\chapter{Liouville 方程}
    \section{Liouville方程}
    如果定义一个概率密度$\rho(\bm{x},\bm{p})$,它满足归一化条件,且处处不小于0. 假设初始条件下在$\bm{x}_0,\bm{p}_0$位置有个体积元$\mathrm{d}\bm{x}_0 \mathrm{d}\bm{p}_0$,跟踪这个体积元经历的轨线,达到$\mathrm{d}\bm{x}_t \mathrm{d}\bm{p}_t$时,在这个体积元的概率应为不变的。这可以理解为,根据Liouville定理,最开始在体积元里面的状态仍然会在初始状态演化后的体积元里。这可以表述为
    \begin{equation*}
        \rho(\bm{x}_t,\bm{p}_t) = \rho(\bm{x}_0,\bm{p}_0)
    \end{equation*}
    它对任意的$t$都成立,则
    \begin{equation*}
        \frac {\mathrm{d}\rho}{\mathrm{d}t} = \frac {\partial \rho}{\partial t} + \frac {\partial \rho}{\partial \bm{x}_t}\dot{\bm{x}}_t  + \frac {\partial \rho}{\partial \bm{p}_t} \dot{\bm{p}}_t = 0
    \end{equation*}
    再利用正则方程,得到
    \begin{equation*}
        - \frac {\partial \rho}{\partial t} = \bigg(\frac {\partial \rho}{\partial \bm{x}_t}\bigg)^\mathrm{T} \frac {\partial H}{\partial \bm{p}_t} - \bigg(\frac {\partial \rho}{\partial \bm{p}_t}\bigg)^\mathrm{T} \frac {\partial H}{\partial \bm{x}_t}
    \end{equation*}
    定义\textbf{Poisson括号}为
    \begin{equation*}
        \{ \rho, H\} = \bigg(\frac {\partial \rho}{\partial \bm{x}_t}\bigg)^\mathrm{T} \frac {\partial H}{\partial \bm{p}_t} - \bigg(\frac {\partial \rho}{\partial \bm{p}_t}\bigg)^\mathrm{T} \frac {\partial H}{\partial \bm{x}_t}
    \end{equation*}
    则有
    \begin{equation*}
        - \frac {\partial \rho}{\partial t} = \{ \rho, H\}
    \end{equation*}
    这也是Liouville定理的一种形式。如果Hamilton函数满足形式
    \begin{equation*}
        H(\bm{x}_t,\bm{p}_t) = \frac 12 \bm{p}_t^\mathrm{T} \bm{M}^{-1} \bm{p}_t + V(\bm{x}_t)
    \end{equation*}
    则有
    \begin{equation*}
        - \frac {\partial \rho}{\partial t} = \bigg(\frac {\partial \rho}{\partial \bm{x}_t}\bigg)^\mathrm{T} \bm{M}^{-1} \bm{p}_t  - \bigg(\frac {\partial \rho}{\partial \bm{p}_t}\bigg)^\mathrm{T} \frac {\partial V}{\partial \bm{x}_t}
    \end{equation*}

    一种常见的分布:\textbf{Boltzmann分布}:
    \begin{equation*}
        \rho(\bm{x},\bm{p}) \propto \mathrm{e}^{-\beta H(\bm{x},\bm{p})}
    \end{equation*}

    如果一个分布满足
    \begin{equation*}
        \frac {\partial \rho}{\partial t} = 0
    \end{equation*}
    则称之为\textbf{稳态分布}。但是即使不是稳态分布,它也会满足对时间的全导数是0。这也是Liouville定理的一个形式。
    研究一个概率密度的时候,有两种方式:一种是研究密度对时间的偏导,看静止空间的概率密度的变化,
    这称为\textbf{Euler图象}。另一种方式是研究密度对时间的劝导,跟踪状态运动的轨线,
    研究这个密度体积元在不同的时间的位置,这称为\textbf{Lagrange图象}。
    \section{求解Liouville方程}
    \subsection{20201012:Euler图象演化概率密度}
    Liouville定理有两种表述形式:
    \begin{equation*}
        -\frac {\partial \rho}{\partial t} = \{ \rho,H \}
    \end{equation*}
    以及 
    \begin{equation*}
        \frac {\mathrm{d}\rho}{\mathrm{d}t} = 0
    \end{equation*}
    第一种形式下,$\rho = \rho(x, p ,t)$, 第二种形式下$\rho = \rho(x_t,p_t,t)$. 分别表示了Euler和Lagrange两种图象。

    回顾描述HCl分子的振动的例子,我们可以用Morse势来描述这个振动:
    \begin{equation*}
        V(x) = D_e (1- \mathrm{e}^{-a(r-r_\mathrm{eq})})^2 = D_e(1-\mathrm{e}^{-ax})^2
    \end{equation*}
    其中有$a>0$, 在平衡位置附近可以使用谐振子近似。写出其Boltzmann分布
    \begin{equation*}
        \rho(x,p,0) = \frac 1Z \mathrm{e}^{-\beta (\frac {p^2}{2m} + \frac 12 m\omega^2 x^2)} 
    \end{equation*}
    由概率密度的归一化,可以得到配分函数的值,这里涉及到Gauss函数的积分
    \begin{align*}
        I &= \int_0^{+\infty} \mathrm{e}^{-ax^2} x^{n} \mathrm{d}x
    \end{align*}
    令$t = ax^2$, 则$\mathrm{d}t = 2ax\mathrm{d}x$
    所以
    \begin{align*}
    I &= \int_0^{+\infty} \mathrm{e}^{-t} \bigg(\frac ta\bigg)^{\frac n2} \frac {\mathrm{d}t}{\sqrt{at}}\\
    &= \frac 1{2a^{\frac {n+1}2}} \int_0^{+\infty} \mathrm{e}^{-t} t^{\frac {n-1}2} \mathrm{d}t\\
    &= \frac {\Gamma(\frac {n+1}2)}{2a^{\frac {n+1}2}}
    \end{align*}
    据此算出配分函数
    \begin{equation*}
        Z = \int \mathrm{e}^{-\beta (\frac {p^2}{2m} + \frac 12 m\omega^2 x^2)} \mathrm{d}x\mathrm{d}p = \frac {2\pi}{\beta \omega}
    \end{equation*}
    从量纲上分析,在配分函数中少了$\mathrm{d}x\mathrm{d}p$的量纲。本质上应该除以$2\pi\hbar$, 相当于对相空间做了量子化。于是
    \begin{equation*}
        Z = \frac 1{\beta \hbar \omega}
    \end{equation*}
    就是无量纲的配分函数。

    回到用Morse势描述HCl的振动的问题,Morse势的常数$a$可以用谐振子近似的$\omega$进行估计。令$x \to 0 $,对$V(x)$在平衡位置附近作Taylor展开,展开到二阶。
    \begin{equation*}
        V(x) = D_e a^2 x^2 + o(x^2)
    \end{equation*}
    它与谐振子近似一致,因此
    \begin{equation*}
        \frac 12 m\omega^2x^2 = D_e a^2 x^2
    \end{equation*}
    于是
    \begin{equation*}
        \omega = \sqrt{\frac {2D_ea^2}m}
    \end{equation*}
    事实上,对双原子分子HCl, 它有6个自由度,3个平动,2个转动,所以我们可以只用振动自由度来描述HCl的分子结构。

    \subsection{20201016:Lagrange图象演化概率密度}
    除了用Euler图象来演化密度以外,也可以用Lagrange图象来演化密度。由
    \begin{equation*}
        \frac {\mathrm{d}}{\mathrm{d}t} \rho(x_t,p_t,t) = 0
    \end{equation*}
    可以得到$t$时刻的概率密度为
    \begin{equation*}
        \rho(x,p,t) = \int \rho(x_0,p_0,0)\delta(x-x_t(x_0,p_0)) \delta(p-p_t(x_0,p_0)) \mathrm{d}x_0\mathrm{d}p_0
    \end{equation*}
    这里引入了$\delta$函数。$\delta$函数满足
    \begin{align*}
        \delta(x-x_0) &= 0, \ \forall \ x \neq x_0\\
        \int_{-\infty}^{+\infty} \delta(x-x_0) \mathrm{d}x &= 1\\
        \int_{-\infty}^{+\infty} f(x)\delta(x-x_0) \mathrm{d}x &= f(x_0)
    \end{align*}
    现在希望给$\delta$函数给一个形式,让它和上面满足的性质自洽:
    可以利用Fourier变换及其逆变换的定义
    \begin{align*}
        \frac 1{\sqrt{2\pi}} \int_{-\infty}^{+\infty} f(x)\mathrm{e}^{\mathrm{i}kx}\mathrm{d}x &= F(k)\\
        \frac 1{\sqrt{2\pi}} \int_{-\infty}^{+\infty} F(k)\mathrm{e}^{-\mathrm{i}kx}\mathrm{d}k &= f(x)
    \end{align*}
    于是有
    \begin{align*}
        f(x_0) &= \frac 1{\sqrt{2\pi}} \int_{-\infty}^{+\infty} \frac 1{\sqrt{2\pi}} \int_{-\infty}^{+\infty} f(x)\mathrm{e}^{\mathrm{i}kx}\mathrm{d}x \mathrm{e}^{-\mathrm{i}kx_0}\mathrm{d}k\\
        &= \frac 1{2\pi} \iint f(x)\mathrm{e}^{\mathrm{i}k(x-x_0)}\mathrm{d}x\mathrm{d}k\\
        &= \frac 1{2\pi} \iint f(x)\mathrm{e}^{\mathrm{i}k(x-x_0)}\mathrm{d}k\mathrm{d}x
    \end{align*}
    于是可以写出$\delta$函数为
    \begin{equation*}
        \delta(x-x_0) = \frac 1{2\pi} \int_{-\infty}^{+\infty} \mathrm{e}^{\mathrm{i}k(x-x_0)}\mathrm{d}k
    \end{equation*}

    某个物理量的期望定义为
    \begin{equation*}
        \langle B(t) \rangle = \int \rho(x,p,t) B(x,p) \mathrm{d}x\mathrm{d}p
    \end{equation*}
    回到用Morse势描述HCl的振动的问题,在这个问题下,初始时刻为Boltzmann分布时,
    \begin{align*}
        \langle x \rangle &= 0\\
        \langle x^2 \rangle &= \frac 1{\beta m \omega^2}\\
        \Delta x &= \sqrt{\langle x^2 \rangle - \langle x \rangle ^2} = \frac 1{\sqrt{\beta m \omega^2}}
    \end{align*}

    \section{Homework}
    \begin{asg}
        Boltzmann分布是否为稳态分布?
    \end{asg}
    \begin{asg}
        第2次作业第2题:构造\ce{H2}分子的Morse势
    \end{asg}
    \begin{asg}
        以Boltzmann分布为初始分布?,在Morse势,Euler图象下演化\ce{H2}的$t$时刻的分布。
    \end{asg}
    \begin{asg}
        以Boltzmann分布为初始分布?,在Morse势,Lagrange图象下演化\ce{H2}的$t$时刻的分布。
    \end{asg}

    \bibliographystyle{plain}
    \bibliography{ref_chp_3}