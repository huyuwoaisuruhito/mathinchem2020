\chapter{Gauss积分}
    \section{20201106:Gauss积分的计算}
        回顾一维Gauss积分的计算:
        \begin{align*}
            I &= \int_0^{+\infty} \mathrm{e}^{-ax^2} x^{n} \mathrm{d}x
        \end{align*}
        令$t = ax^2$, 则$\mathrm{d}t = 2ax\mathrm{d}x$
        所以
        \begin{align*}
        I &= \int_0^{+\infty} \mathrm{e}^{-t} \bigg(\frac ta\bigg)^{\frac n2} \frac {\mathrm{d}t}{\sqrt{at}}\\
        &= \frac 1{2a^{\frac {n+1}2}} \int_0^{+\infty} \mathrm{e}^{-t} t^{\frac {n-1}2} \mathrm{d}t\\
        &= \frac {\Gamma(\frac {n+1}2)}{2a^{\frac {n+1}2}}
        \end{align*}

        利用一维Gauss积分的计算结果可以计算多维的Gauss积分。比如计算
        \begin{equation*}
            I = \int \bm{x}^\mathrm{T}\bm{Bx} \mathrm{e}^{-\bm{x}^\mathrm{T}\bm{Ax}}\mathrm{d}\bm{x}
        \end{equation*}
        其中,$\bm{A}$为正定的实对称矩阵。首先将$\bm{A}$对角化,得到
        \begin{equation*}
            \bm{T}^\mathrm{T} \bm{AT} = \bm{D}
        \end{equation*}
        其中$\bm{T}$为正交矩阵、$\bm{D}$为对角矩阵。所以
        \begin{equation*}
            I = \int \bm{x}^\mathrm{T}\bm{Bx} \mathrm{e}^{-\bm{x}^\mathrm{T}\bm{TDT}^\mathrm{T}\bm{x}}\mathrm{d}\bm{x}
        \end{equation*}
        作换元$\bm{y} = \bm{T}^\mathrm{T}\bm{x}$,则有
        \begin{equation*}
            \mathrm{d}\bm{y} = \det \bm{T}^\mathrm{T}\mathrm{d}\bm{x} = \mathrm{d}\bm{x}
        \end{equation*}
        原积分化为
        \begin{align*}
            I = \int \bm{y}^\mathrm{T}\bm{T}^\mathrm{T}\bm{BTy} \mathrm{e}^{-\bm{y}^\mathrm{T}\bm{D}\bm{y}}\mathrm{d}\bm{y}
        \end{align*}
        令$\bm{E = T}^\mathrm{T}\bm{BT}$, 于是
        \begin{align*}
            I &= \int \bm{y}^\mathrm{T}\bm{Ey} \mathrm{e}^{-\bm{y}^\mathrm{T}\bm{D}\bm{y}}\mathrm{d}\bm{y}\\
            &= \int \sum_i \sum_j E_{ij} y_i y_j \mathrm{e}^{-\sum_k d_k y_k^2} \prod_k \mathrm{d}y_k
        \end{align*}
        可以通过分离变量将各个积分分开,显然,$i \neq j$的项都是奇函数对全空间的积分,得到的结果为0,只有$i=j$的项会有贡献。于是积分化为
        \begin{align*}
            I &= \int \sum_i E_{ii} y_i^2 \mathrm{e}^{-\sum_k d_k y_k^2} \prod_k \mathrm{d}y_k\\
            &= \prod_k \sqrt{\frac {\pi}{d_k}} \sum_i \frac {E_{ii}}{2d_i}\\
            &= \frac {\pi^{\frac n2}}{2\sqrt{\det{\bm{D}}}} \mathrm{Tr} (\bm{ED}^{-1})
        \end{align*}
        而
        \begin{align*}
            \det{\bm{D}} = \det{\bm{A}}
        \end{align*}
        并且
        \begin{align*}
            \mathrm{Tr}(\bm{ED}^{-1}) = \mathrm{Tr}(\bm{T}^\mathrm{T}\bm{BT}\bm{T}^\mathrm{T} \bm{A}^{-1}\bm{T})
            = \mathrm{Tr}(\bm{T}^\mathrm{T}\bm{BA}^{-1}\bm{T})
            = \mathrm{Tr}(\bm{BA}^{-1})
        \end{align*}
        所以
        \begin{align*}
            I = \frac {\pi^{\frac n2}}{2\sqrt{\det{\bm{A}}}} \mathrm{Tr} (\bm{BA}^{-1})
        \end{align*}

        接下来尝试计算
        \begin{align*}
            \bm{I} = \int \bm{xx}^\mathrm{T} \mathrm{e}^{-\bm{x}^\mathrm{T}\bm{Ax}}\mathrm{d}\bm{x}
        \end{align*}
        用相同的还原方法得到
        \begin{align*}
            \bm{I} = \int \bm{Ty}\bm{y}^\mathrm{T}\bm{T}^\mathrm{T} \mathrm{e}^{-\bm{y}^\mathrm{T}\bm{Dy}}\mathrm{d}\bm{y}
        \end{align*}
        计算每个元素
        \begin{align*}
            I_{ij} &= \int (\bm{Ty})_{i}(\bm{y}^\mathrm{T}\bm{T}^\mathrm{T})_{j} \mathrm{e}^{-\bm{y}^\mathrm{T}\bm{Dy}}\mathrm{d}\bm{y}\\
            &= \int \sum_k T_{ik}y_k \sum_l T_{jl} \mathrm{e}^{-\bm{y}^\mathrm{T}\bm{Dy}}\mathrm{d}\bm{y}\\
            &= \int \sum_{k,l} T_{ik}T_{jl} y_ky_l \mathrm{e}^{-\bm{y}^\mathrm{T}\bm{Dy}}\mathrm{d}\bm{y}
        \end{align*}
        同样地,只有在$k=l$时才有贡献,故
        \begin{align*}
            I_{ij} &= \int \sum_k T_{ik}T_{jk}y_k^2 \mathrm{e}^{-\bm{y}^\mathrm{T}\bm{Dy}}\mathrm{d}\bm{y}\\
            &= \prod_l \sqrt{\frac {\pi}{d_l}} \sum_k T_{ik}T_{jk} \frac 1{2d_k}\\
            &= \frac {\pi^{\frac n2}}{2 \sqrt{\det{\bm{D}}}} (\bm{TD}^{-1}\bm{T}^\mathrm{T})_{ij}\\
            &= \frac {\pi^{\frac n2}}{2 \sqrt{\det{\bm{A}}}} \bm{A}^{-1}_{ij}
        \end{align*}
        因此,
        \begin{align*}
            \bm{I} = \frac {\pi^{\frac n2}}{2 \sqrt{\det{\bm{A}}}} \bm{A}^{-1}
        \end{align*}
        \begin{asg}
            第4次作业第3题:Gauss积分的计算
        \end{asg}
        \bibliographystyle{plain}
        \bibliography{ref_chp_6}