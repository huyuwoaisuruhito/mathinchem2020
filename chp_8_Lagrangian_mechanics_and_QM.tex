\chapter{Lagrange力学和量子力学的路径积分形式}
    \section{作用量与Lagrange力学}
        \subsection{最小作用量原理}
        \subsection{Hamilton-Jacobi方程}
    \section{量子力学的路径积分形式}
        如果求出传播子
        \begin{equation*}
            \langle x_0 | \mathrm{e}^{-\frac {\mathrm{i}\hat{H}t}{\hbar}} | y_0 \rangle
        \end{equation*}
        那么就可以求解含时Schrodinger方程,这时就将研究对象从波函数变为传播子。所以现在需要求解传播子。
        首先研究$\mathrm{e}^{\lambda \hat{A}}\mathrm{e}^{\lambda \hat{B}}$和$\mathrm{e}^{\lambda (\hat{A}+\hat{B})}$的关系。应有
        \begin{equation*}
            \mathrm{e}^{\lambda \hat{A}} \mathrm{e}^{\lambda \hat{B}} = \mathrm{e}^{\lambda (\hat{A}+\hat{B}) + \frac 12 \lambda^2 [\hat{A},\hat{B}] + O(\lambda^2)}
        \end{equation*}
        如果$\lambda \to 0$, 可以忽略二阶无穷小量,则有
        \begin{equation*}
            \mathrm{e}^{\lambda \hat{A}} \mathrm{e}^{\lambda \hat{B}} = \mathrm{e}^{\lambda (\hat{A}+\hat{B})}
        \end{equation*}
        将时间平均分为$N$份,令$\lambda = \frac tN$, 并令$N \to \infty$.所以
        \begin{equation*}
            \mathrm{e}^{\frac tN (-\frac {\mathrm{i}}{\hbar}) \hat{K}} \mathrm{e}^{\frac tN (-\frac {\mathrm{i}}{\hbar}) \hat{V}} =\mathrm{e}^{\frac tN (-\frac {\mathrm{i}}{\hbar}) \hat{H}} 
        \end{equation*}
        代入传播子,得到
        \begin{align*}
            \langle x_0 | \mathrm{e}^{-\frac {\mathrm{i}\hat{H}t}{N\hbar}} | y_0 \rangle &= \langle x | \mathrm{e}^{\frac tN (-\frac {\mathrm{i}}{\hbar}) \frac {\hat{p}^2}{2m}}
            \mathrm{e}^{\frac tN (-\frac {\mathrm{i}}{\hbar}) \hat{V}} |y \rangle\\
            &= \langle x | \mathrm{e}^{\frac tN (-\frac {\mathrm{i}}{\hbar}) \frac {\hat{p}^2}{2m}} |y \rangle \mathrm{e}^{\frac tN (-\frac {\mathrm{i}}{\hbar}) V(y)}\\
            &= \int \langle x | \mathrm{e}^{\frac tN (-\frac {\mathrm{i}}{\hbar}) \frac {\hat{p}^2}{2m}} |p \rangle \langle p |y \rangle \mathrm{d}p \times \mathrm{e}^{\frac tN (-\frac {\mathrm{i}}{\hbar}) V(y)}\\
            &= \int \langle x|p \rangle \langle p|y \rangle \mathrm{e}^{\frac tN (-\frac {\mathrm{i}}{\hbar}) \frac {p^2}{2m}} \mathrm{d}p \times \mathrm{e}^{\frac tN (-\frac {\mathrm{i}}{\hbar}) V(y)}\\
            &= \frac 1{2\pi \hbar} \int \mathrm{e}^{\frac {i(x-y)p}{\hbar}} \mathrm{e}^{\frac tN (-\frac {\mathrm{i}}{\hbar}) \frac {p^2}{2m}} \mathrm{d}p \times \mathrm{e}^{\frac tN (-\frac {\mathrm{i}}{\hbar}) V(y)}
        \end{align*}
        根据Gauss积分
        \begin{equation*}
            \int_{-\infty}^{+\infty} \mathrm{e}^{-ax^2+bx} \mathrm{d}x = \sqrt{\frac {\pi}a} \mathrm{e}^{-\frac {b^2}{4a}}
        \end{equation*}
        由此得到传播子为
        \begin{align*}
            \langle x_0 | \mathrm{e}^{-\frac {\mathrm{i}\hat{H}t}{N\hbar}} | y_0 \rangle &= \sqrt{\frac {mN}{2\pi\mathrm{i} \hbar t}} \mathrm{e}^{-\mathrm{i}\frac {mN(x-y)^2}{2\hbar t}}\mathrm{e}^{-\frac {\mathrm{i}t}{N\hbar} V(y)}
        \end{align*}
        前两项来自于动能算符,第三项来自于势能算符。动能算符和势能算符虽然不对易,但是在$N \to \infty$时可以得到这个结果。
        自由粒子体系的势能为0,所以可以不需要把时间分成$N$份,而是直接对整个传播子来计算。把$V=0,N=1$代入上式,即得到
        \begin{align*}
            \langle x_0 | \mathrm{e}^{-\frac {\mathrm{i}\hat{H}t}{\hbar}} |y_0 \rangle &= \sqrt{\frac {m}{2\pi\mathrm{i} \hbar t}} \mathrm{e}^{-\mathrm{i}\frac {m(x-y)^2}{2\hbar t}}
        \end{align*}
        如果推广到$F$维体系,则有
        \begin{align*}
            \langle \bm{x_0} | \mathrm{e}^{-\frac {\mathrm{i}\hat{H}t}{\hbar}} |\bm{y_0} \rangle &= (\frac {1}{2\pi\mathrm{i} \hbar t})^{\frac F2} |\bm{M}|^{\frac 12} \mathrm{e}^{-\mathrm{i}\frac {\bm{(x-y)}^{\mathrm{T}} \bm{M (x-y)}}{2\hbar t}}
        \end{align*}
        可以将传播子写成
        \begin{align*}
            \langle y_0 | \mathrm{e}^{-\frac {\mathrm{i}\hat{H}t}{\hbar}} |x_0 \rangle &= C(t) \mathrm{e}^{\frac {\mathrm{i}S(x(t))}{\hbar}}
        \end{align*}
        在经典情况下写出作用量
        \begin{align*}
            S(x(t)) = \int_0^t \mathcal{L}(x,\dot{x},t') \mathrm{d}t' = \frac 12 \int_0^t m\dot{x}^2 \mathrm{d}t'
        \end{align*}
        Lagrange函数会满足Euler-Lagrange方程,而对于自由粒子,Lagrange函数不显含坐标,所以
        \begin{align*}
            \frac {\mathrm{d}}{\mathrm{d}t} (m \dot x) = 0
        \end{align*}
        由此可见,速度不随时间变化,且
        \begin{align*}
            \dot{x} = \frac {y_0-x_0}t
        \end{align*}
        所以,上述作用量积分的结果为
        \begin{align*}
            S = \frac {m(y_0 - x_0)^2}{2t}
        \end{align*}
        显然地,这个结果代入上面写出的传播子表达式相吻合。现在希望能把$C(t)$求出。给定初始条件
        \begin{equation*}
            t \to 0,~~~~~\langle y_0|x_0\rangle = \delta(y_0-x_0)
        \end{equation*}
        计算出
        \begin{align*}
            \int_{-\infty}^{+\infty} \mathrm{e}^{\frac {\mathrm{i}}{\hbar} \frac {m(y_0 - x_0)^2}{2t}}\mathrm{d}y_0 = C(t) \sqrt{\frac {2\pi\mathrm{i}\hbar t}m}
        \end{align*}
        于是 
        \begin{align*}
            C(t) = \sqrt{\frac m{2\pi\mathrm{i}\hbar t}} D(t)
        \end{align*}
        其中$D(0) = 1$. 现在希望证明$D(t) = 1$.
        计算
        \begin{align*}
            -\mathrm{i}\hbar \frac {\partial}{\partial t} \langle y_0 | \mathrm{e}^{-\frac {\mathrm{i}\hat{H}t}{\hbar}} |x_0 \rangle =  \langle y_0 | \hat{H} \mathrm{e}^{-\frac {\mathrm{i}\hat{H}t}{\hbar}} |x_0 \rangle
            = -\frac {\hbar^2}{2m} \frac {\partial^2}{\partial y_0^2} \langle y_0 | \mathrm{e}^{-\frac {\mathrm{i}\hat{H}t}{\hbar}} |x_0 \rangle
        \end{align*}
        \begin{asg}
            第8次作业第四题
        \end{asg}
        如果不是自由体系,则使用\textbf{多边折线方案}。
    \section{路径积分分子动力学}
        假设已经得到传播子
        \begin{equation*}
            \langle x | \mathrm{e}^{-\frac {i \hat{H}t}{\hbar}} |y \rangle = \sqrt{\frac m{2\pi \mathrm{i}\hbar t}} \mathrm{e}^{\mathrm{i}\frac {m(x-y)^2}{2t\hbar}}
        \end{equation*}
        路径积分是在空间中连接所有$x,y$的路径都要进行考虑。所以,传播子是对所有路径求和
        \begin{align*}
            \langle x | \mathrm{e}^{-\frac {i \hat{H}t}{\hbar}} = \sum_{\mathrm{all~paths}} C_t \mathrm{e}^{\mathrm{i}S_t\hbar}
        \end{align*}
        其中 
        \begin{align*}
            C_t &= \sqrt{\frac m{2\pi \mathrm{i}\hbar t}} \\
            S_t &= \int_0^t \mathcal{L}(x,\dot{x},t') \mathrm{d}t' = \int_0^t (\frac 12 m \dot{x}^2 - V(x)) \mathrm{d}t'
        \end{align*}

        量子体系下的Boltzmann分布为
        \begin{equation*}
            \mathrm{e}^{-\beta \hat{H}} = \sum_n \mathrm{e}^{-\beta E_n} |\phi_n \rangle \langle \phi_n|
        \end{equation*}
        利用了Schodinger方程
        \begin{equation*}
            \hat{H} |\phi_n \rangle = E_n |\phi_n \rangle
        \end{equation*}
        类似地,可以作和
        \begin{align*}
            \mathrm{e}^{-\frac {\mathrm{i}\hat{H}t}{\hbar}} = \sum_n \mathrm{e}^{-\frac {\mathrm{i}E_n t}{\hbar}} |\phi_n \rangle \langle \phi_n|
        \end{align*}
        上述两个式子可以对应起来,区别在于第二个方程中的时间是虚数,称为\textbf{虚时间}。对应关系为
        \begin{equation*}
            t = -\mathrm{i}\hbar \beta
        \end{equation*}
        显然地,高温对应虚时间的短时,低温对应虚时间的长时。同样可求出虚时间下的传播子
        \begin{align*}
            \langle x|\mathrm{e}^{-\beta \hat{H}}|y\rangle
        \end{align*}
        求出配分函数
        \begin{align*}
            Z &= \mathrm{Tr}(\mathrm{e}^{-\beta \hat{H}}) \\
            &= \sum_n \langle n| \mathrm{e}^{-\beta \hat{H}}|n\rangle \\
            &= \int \sum_n \langle n| \mathrm{e}^{-\beta \hat{H}}|x \rangle \langle x|n\rangle \mathrm{d}x\\
            &= \int \langle x|\sum_n|n \rangle \langle n| \mathrm{e}^{-\beta \hat{H}}|x\rangle \mathrm{d}x\\
            &= \int \langle x|\mathrm{e}^{-\beta \hat{H}}|x\rangle \mathrm{d}x
        \end{align*}
        在路径积分的语言下,可以放弃态的概念,也不需要有波函数,只要有传播子,就有配分函数,也就有了所有的热力学函数。要求这个积分中的传播子,将$\beta$分为$N$份.有
        \begin{align*}
            \langle x|\mathrm{e}^{-\beta \hat{H}}|x\rangle  = \int \langle x_0|\mathrm{e}^{-\frac {\beta \hat{H}}N} |x_1 \rangle ... \langle x_{n-2}|\mathrm{e}^{-\frac {\beta \hat{H}}N} |x_{n-1} \rangle \langle x_{n-1}|\mathrm{e}^{-\frac {\beta \hat{H}}N} |x_N \rangle \prod_{i=1}^{N-1}\mathrm{d}x_i
        \end{align*}
        其中$x_0=x_N = x$. 如果$N \to \infty$, 则可以把动能项和势能项分开。用和之前传播子计算类似的方法得到
        \begin{align*}
            \langle x_j|\mathrm{e}^{-\Delta \beta \hat{H}} |x_{j+1} \rangle &= \sqrt{\frac m{2\pi \hbar^2 \Delta \beta}} \mathrm{e}^{-\frac {m(x_j- x_{j+1})^2}{2\hbar^2 \Delta \beta}} \mathrm{e}^{-\Delta \beta V(x_{j+1})}
        \end{align*}
        定义$\omega_N^2 = \frac N{\hbar^2 \Delta \beta^2} = \frac N{\hbar^2 \beta^2}$,则有
        \begin{align*}
            \langle x_j|\mathrm{e}^{-\Delta \beta \hat{H}} |x_{j+1} \rangle &= \sqrt{\frac m{2\pi \hbar^2 \Delta \beta}} \mathrm{e}^{-\frac {\beta m\omega_N^2(x_j- x_{j+1})^2}{2}} \mathrm{e}^{-\Delta \beta V(x_{j+1})}
        \end{align*}
        代入得到 
        \begin{align*}
            \langle x|\mathrm{e}^{-\beta \hat{H}}|x\rangle  = \bigg(\frac {mN}{2\pi \hbar^2 \beta}\bigg)^{\frac N2} \int \mathrm{e}^{-\sum_{j=0}^{N-1} \frac {\beta}2 m \omega_N^2 (x_{j+1} - x_j)^2} \mathrm{e}^{-\Delta \beta \sum_{j=0}^{N-1}V(x_j)}\prod_{i=1}^{N-1}\mathrm{d}x_i
        \end{align*}
        这可以看作$N$个点组成的环两两用弹簧连接,且每个点都额外受外力作用。它可以写成
        \begin{align*}
            \langle x|\mathrm{e}^{-\beta \hat{H}}|x\rangle  = \int C(N) \mathrm{e}^{-\beta V_{\mathrm{eff}}(\bm{x})} \mathrm{d}\bm{x}
        \end{align*}
        积分不太容易做,可以在这里插入一个关于“动量”的积分
        \begin{align*}
            \langle x|\mathrm{e}^{-\beta \hat{H}}|x\rangle  &= \int D(N) \mathrm{e}^{-\beta V_{\mathrm{eff}}(\bm{x})} \mathrm{d}\bm{x} \int \mathrm{d}\bm{p} \mathrm{e}^{-\frac {\beta}2 \bm{p}^{\mathrm{T}}\bm{M}^{-1} \bm{p}}\\
            &= \int D(N) \mathrm{e}^{-\beta H_{\mathrm{eff}}(\bm{x,p})} \mathrm{d}\bm{x} \mathrm{d}\bm{p}
        \end{align*}
        如果$N=1$,那么就是经典统计力学的结果。这体现了量子力学和经典统计力学的同构。

        如果关心含时Schodinger方程:
        \begin{equation*}
            \mathrm{i}\hbar \frac {\partial}{\partial t} | \psi(t) \rangle = \hat{H}|\psi(t) \rangle
        \end{equation*}
        将这个改成对于$\beta$的方程:
        \begin{align*}
            -\frac {\partial}{\partial \beta} |\psi(\beta) \rangle = \hat{H} |\psi(\beta) \rangle
        \end{align*}

        任意给定一个初态,可以写成Hamilton函数的本征函数的线性组合
        \begin{align*}
            |\psi(0) \rangle = \sum_n c_n |\phi_n\rangle
        \end{align*}
        并且
        \begin{align*}
            |\psi(\beta) \rangle &= \mathrm{e}^{-\beta \hat{H}} |\phi(0) \rangle = \sum_n c_n\mathrm{e}^{-\beta E_n}|\phi_n \rangle = \mathrm{e}^{-\beta E_0} \sum_n c_n \mathrm{e}^{-\beta(E_n-E_0)} |\phi_n \rangle
        \end{align*}
        因此,当$\beta \to \infty$时,得到的就是基态。基于此开发出了\textbf{量子Monte-Carlo算法}
        如果要求第一激发态,只需求出系数,把基态从初始条件中减去,得到的新的最低的能量对应的态就是第一激发态。
        \bibliographystyle{plain}
        \bibliography{ref_chp_8}