\chapter{Liouville 定理}
    匀变速直线运动,应当有
    \begin{align*}
        x(t) &= x(0) + vt + \frac 12 at^2\\
        &= x(0) + \dot{x}t + \frac 12 \ddot{x}t^2
    \end{align*}
    这相当于位置对时间作了Taylor展开,展开到二阶。但是为什么只考虑前两阶,而不考虑后面的项呢?
    可以这样考虑:在给定了Hamilton函数的情形下,正则方程最多只涉及到对时间的二阶导数,
    最终解出位置对时间的函数,以及动量对时间的函数只有两个待定常数,因此只用位置和动量初始的条件。
    \footnote{
        这里给出的是笔记书写者的看法。为什么牛顿方程是二阶常微分方程,修改者认为这是由经典力学的
        \textbf{决定性}导致的。力学系统的位置和位置对于时间的导数可以唯一决定力学系统今后的状态,
        如果在位形空间中列出运动方程(Newton方程、Euler-Lagrange方程),那必然是二阶微分方程
        (此时微分方程解的存在唯一性定理与经典力学的决定性相容);如果在相空间
    }
    \section{20200928:相空间不同时刻体积元的关系}
    现在开始研究一个多维体系,它的位置和动量分别不是一个标量,而是一个向量$\bm{x}, \bm{p}$. 如果系统的在$t$时刻的状态$(\bm{x}_t,\bm{p}_t)$对应一个相空间中的\textbf{体积元}:$\mathrm{d}\bm{x}_t\mathrm{d}\bm{p}_t$。如果给定初始条件$(\bm{x}_0,\bm{p}_0)$, 希望在正则方程成立的条件下,能够确定0时刻和$t$时刻的相空间体积元的关系。这个问题可以等效地理解为,将初始条件产生一个很小的偏差$(\mathrm{d}\bm{x}_0,\mathrm{d}\bm{p}_0)$,要求在$t$时刻的偏差和初始条件的关系。

    这实际上给出了两种研究问题的办法:一种是参考系不动,一种是参考系随着时间跟踪系统在相空间中的轨线进行运动。

    对于任意个不显含时间的函数$f(\bm{x}_t,\bm{p}_t)$, 它和$f(\bm{x}_0,\bm{p}_0)$的关系为:
    \begin{align*}
        \int f(\bm{x}_t,\bm{p}_t) \mathrm{d}\bm{x}_t\mathrm{d}\bm{p}_t = \int f(\bm{x}_0,\bm{p}_0)\bigg|\frac {\partial (\bm{x}_t, \bm{p}_t)}{\partial (\bm{x}_0,\bm{p}_0)}\bigg| \mathrm{d}\bm{x}_0 \mathrm{d}\bm{p}_0
    \end{align*}
    由此可知,算出Jacobi行列式的值是非常重要的。Jacobi行列式的对应矩阵写为
    \begin{align*}
        \begin{pmatrix}
            \frac {\partial \bm{x}_t}{\partial \bm{x}_0} & \frac {\partial \bm{x}_t}{\partial \bm{p}_0}\\
            \frac {\partial \bm{p}_t}{\partial \bm{x}_0} & \frac {\partial \bm{p}_t}{\partial \bm{p}_0}
        \end{pmatrix}
    \end{align*}

    我们可以把$t$时刻的状态写成初始条件的函数:
    \begin{align*}
        \bm{x}_t &= \bm{x}_t(\bm{x}_0, \bm{p}_0)\\
        \bm{p}_t &= \bm{p}_t(\bm{x}_0, \bm{p}_0)
    \end{align*}
    如果初始状态偏离$(\mathrm{d}\bm{x}_0,\mathrm{d}\bm{p}_0)$,那么
    \begin{align*}
        \bm{x}_t(\bm{x}_0+\mathrm{d}\bm{x}_0, \bm{p}_0+\mathrm{d}\bm{p}_0) &= \bm{x}_t(\bm{x}_0, \bm{p}_0) + \frac {\partial \bm{x}_t}{\partial \bm{x_0}} \mathrm{d}x_0 + \frac {\partial \bm{x}_t}{\partial \bm{p}_0} \mathrm{d}p_0\\
        \bm{p}_t(\bm{x}_0+\mathrm{d}\bm{x}_0, \bm{p}_0+\mathrm{d}\bm{p}_0) &= \bm{p}_t(\bm{x}_0, \bm{p}_0) + \frac {\partial \bm{p}_t}{\partial \bm{x_0}} \mathrm{d}x_0 + \frac {\partial \bm{p}_t}{\partial \bm{p}_0} \mathrm{d}p_0
    \end{align*}
    此处只考虑Taylor展开到一阶的结果。或者写成
    \begin{align*}
        \mathrm{d}\bm{x}_t &= \frac {\partial \bm{x}_t}{\partial \bm{x_0}} \mathrm{d}x_0 + \frac {\partial \bm{x}_t}{\partial \bm{p}_0} \mathrm{d}p_0\\
        \mathrm{d}\bm{p}_t &= \frac {\partial \bm{p}_t}{\partial \bm{x_0}} \mathrm{d}x_0 + \frac {\partial \bm{p}_t}{\partial \bm{p}_0} \mathrm{d}p_0
    \end{align*}
    矩阵没有办法直接求出来,我们尝试对时间求导。
    \begin{align*}
        \frac {\mathrm{d}}{\mathrm{d}t} \bigg(\frac {\partial \bm{x}_t}{\partial \bm{x}_0}\bigg)_{\bm{p}_0} &= \bigg(\frac {\partial}{\partial \bm{x}_0} \frac {\mathrm{d}}{\mathrm{d}t} \bm{x}_t\bigg)_{\bm{p}_0} = \bigg(\frac {\partial}{\partial \bm{x}_0} \bigg(\frac {\partial H}{\partial \bm{p}_t}\bigg)_{\bm{x}_t}\bigg)_{\bm{p}_0} = \bigg(\frac {\partial^2 H}{\partial \bm{x}_t \partial \bm{p}_t}\bigg) \bigg(\frac {\partial \bm{x}_t}{\partial \bm{x}_0}\bigg)_{\bm{p}_0}+ \bigg(\frac {\partial^2H}{\partial \bm{p}_t^2}\bigg)_{\bm{x}_t} \bigg(\frac {\partial \bm{p}_t}{\partial \bm{x}_0}\bigg)_{\bm{p}_0}\\
        \frac {\mathrm{d}}{\mathrm{d}t} \bigg(\frac {\partial \bm{x}_t}{\partial \bm{p}_0}\bigg)_{\bm{x}_0} &= \bigg(\frac {\partial}{\partial \bm{p}_0} \frac {\mathrm{d}}{\mathrm{d}t} \bm{x}_t\bigg)_{\bm{x}_0} = \bigg(\frac {\partial}{\partial \bm{p}_0} \bigg(\frac {\partial H}{\partial \bm{p}_t}\bigg)_{\bm{x}_t}\bigg)_{\bm{x}_0} = \bigg(\frac {\partial^2 H}{\partial \bm{x}_t \partial \bm{p}_t}\bigg) \bigg(\frac {\partial \bm{x}_t}{\partial \bm{p}_0}\bigg)_{\bm{x}_0} + \bigg(\frac {\partial^2H}{\partial \bm{p}_t^2}\bigg)_{\bm{x}_t} \bigg(\frac {\partial \bm{p}_t}{\partial \bm{p}_0}\bigg)_{\bm{x}_0}\\
        \frac {\mathrm{d}}{\mathrm{d}t} \bigg(\frac {\partial \bm{p}_t}{\partial \bm{x}_0}\bigg)_{\bm{p}_0} &= \bigg(\frac {\partial}{\partial \bm{x}_0} \frac {\mathrm{d}}{\mathrm{d}t} \bm{p}_t\bigg)_{\bm{p}_0} = -\bigg(\frac {\partial}{\partial \bm{x}_0} \bigg(\frac {\partial H}{\partial \bm{x}_t}\bigg)_{\bm{p}_t}\bigg)_{\bm{p}_0} = -\bigg(\frac {\partial^2 H}{\partial \bm{x}_t^2}\bigg)_{\bm{p}t} \bigg(\frac {\partial \bm{x}_t}{\partial \bm{x}_0}\bigg)_{\bm{p}_0}- \bigg(\frac {\partial^2H}{\partial \bm{p}_t \partial \bm{x}_t}\bigg) \bigg(\frac {\partial \bm{p}_t}{\partial \bm{x}_0}\bigg)_{\bm{p}_0}\\
        \frac {\mathrm{d}}{\mathrm{d}t} \bigg(\frac {\partial \bm{p}_t}{\partial \bm{p}_0}\bigg)_{\bm{x}_0} &= \bigg(\frac {\partial}{\partial \bm{p}_0} \frac {\mathrm{d}}{\mathrm{d}t} \bm{p}_t\bigg)_{\bm{x}_0} = -\bigg(\frac {\partial}{\partial \bm{p}_0} \bigg(\frac {\partial H}{\partial \bm{x}_t}\bigg)_{\bm{p}_t}\bigg)_{\bm{x}_0} = -\bigg(\frac {\partial^2 H}{\partial \bm{x}_t^2}\bigg)_{\bm{p}t} \bigg(\frac {\partial \bm{x}_t}{\partial \bm{p}_0}\bigg)_{\bm{x}_0}- \bigg(\frac {\partial^2H}{\partial \bm{p}_t \partial \bm{x}_t}\bigg) \bigg(\frac {\partial \bm{p}_t}{\partial \bm{p}_0}\bigg)_{\bm{x}_0}
    \end{align*}
    由此可以得到
    \begin{align*}
        \frac {\mathrm{d}}{\mathrm{d}t}
        \begin{pmatrix}
            \frac {\partial \bm{x}_t}{\partial \bm{x}_0} & \frac {\partial \bm{x}_t}{\partial \bm{p}_0}\\
            \frac {\partial \bm{p}_t}{\partial \bm{x}_0} & \frac {\partial \bm{p}_t}{\partial \bm{p}_0}
        \end{pmatrix}
        =
        \begin{pmatrix}
            \frac {\partial^2 H}{\partial \bm{x}_t \partial \bm{p}_t} & (\frac {\partial^2 H}{\partial \bm{p}_t^2})_{\bm{x}_t}\\
            -(\frac {\partial^2 H}{\partial \bm{x}_t^2})_{\bm{p}_t} & - \frac {\partial^2 H}{\partial \bm{x}_t \partial \bm{p}_t}
        \end{pmatrix}
        \begin{pmatrix}
            \frac {\partial \bm{x}_t}{\partial \bm{x}_0} & \frac {\partial \bm{x}_t}{\partial \bm{p}_0}\\
            \frac {\partial \bm{p}_t}{\partial \bm{x}_0} & \frac {\partial \bm{p}_t}{\partial \bm{p}_0}
        \end{pmatrix}
    \end{align*}
    将此处的Jacobi矩阵称为\textbf{稳定性矩阵},其含义是如果系统初始时刻状态变化很小,那么$t$时刻的变化也很小。
    \begin{asg}
        第1次作业第3题:Liouville定理的证明
    \end{asg}

    \section{20201009:Liouville定理}
    设矩阵
    \begin{equation*}
        \bm{A} = 
        \begin{pmatrix}
            a_{11} & \cdots & a_{1n}\\
            \vdots & & \vdots\\
            a_{n1} & \cdots & a_{nn}\\
            \end{pmatrix}
    \end{equation*}
    它的行列式为
    \begin{equation*}
        \det{\bm{A}} = \sum_{j=1}^n (-1)^{i+j} a_{ij} \bm{A}_{ij}^*, \ \forall \ i
    \end{equation*}
    其中,$\bm{A}_{ij}^*$表示$a_{ij}$的代数余子式。定义$\bm{A}$的\textbf{伴随矩阵}$\bar{\bm{A}}$为
    \begin{equation*}
        \bar{\bm{A}}_{ij} = \bm{A}_{ji}^* 
    \end{equation*}
    矩阵的逆矩阵为
    \begin{equation*}
        \bm{A}^{-1} = \frac 1{\det{\bm{A}}} \bm{\bar{A}}
    \end{equation*}
    对行列式的求导并不是对每个元素求导再求行列式,而是依照下列方法:
    \begin{align*}
        \frac {\mathrm{d}}{\mathrm{d}t} \det{\bm{A}} = \sum_i \det{\tilde{\bm{A}_i}}
    \end{align*}
    其中,$\bm{A}_i$是只对第$i$行的所有元素对时间求导,其他元素不变得到的矩阵。进一步得到
    \begin{align*}
        \frac {\mathrm{d}}{\mathrm{d}t} \det{\bm{A}} &= \sum_i \det{\tilde{\bm{A}_i}} = \sum_{i=1}^n \sum_{j=1}^n \frac {\mathrm{d}a_{ij}}{\mathrm{d}t} \bm{A}_{ij}^*\\
        &= \mathrm{Tr} \bigg(\frac {\mathrm{d}\bm{A}}{\mathrm{d}t} \bar{\bm{A}}\bigg) = \mathrm{Tr} \bigg(\frac {\mathrm{d}\bm{A}}{\mathrm{d}t} \bm{A}^{-1}\bigg) \det{\bm{A}}
    \end{align*}
    将两边同时除以$\bm{A}$的行列式,得到
    \begin{equation*}
        \frac {\mathrm{d}}{\mathrm{d}t} \ln{\det{\bm{A}}} = \mathrm{Tr} \bigg(\frac {\mathrm{d}\bm{A}}{\mathrm{d}t} \bm{A}^{-1}\bigg)
    \end{equation*}
    如果
    \begin{equation*}
        \frac {\mathrm{d}}{\mathrm{d}t} \bm{A} = \bm{MA}
    \end{equation*}
    就有
    \begin{equation*}
        \frac {\mathrm{d}}{\mathrm{d}t} \ln{\det{\bm{A}}} = \mathrm{Tr}\ \bm{M}
    \end{equation*}
    对于上一节讲的Jacobi矩阵,有
    \begin{equation*}
        \bm{M} = 
        \begin{pmatrix}
        \frac {\partial^2 H}{\partial \vec{x}_t \partial \vec{p}_t} & (\frac {\partial^2 H}{\partial \vec{p}_t^2})_{\vec{x}_t}\\
        -(\frac {\partial^2 H}{\partial \vec{x}_t^2})_{\vec{p}_t} & - \frac {\partial^2 H}{\partial \vec{x}_t \partial \vec{p}_t}
        \end{pmatrix}
    \end{equation*}
    显然这个矩阵的迹为0,所以
    \begin{equation*}
        \frac {\mathrm{d}}{\mathrm{d}t} \det \bigg|\frac {\partial (\bm{x}_t,\bm{p}_t)}{\partial (\bm{x}_0,\bm{p}_0)} \bigg| = 0
    \end{equation*}
    但初始时刻显然Jacobi行列式为1,所以Jacobi行列式一直为1,就有
    \begin{equation*}
        \mathrm{d}\bm{x}_t \mathrm{d}\bm{p}_t = \mathrm{d}\bm{x}_0 \mathrm{d}\bm{p}_0
    \end{equation*}

    这个结论称为\textbf{Liouville定理}。注意到这个结论的推导只用到了正则方程,只要正则方程成立,这个结论就成立。

    如果定义一个概率密度$\rho(\bm{x},\bm{p})$,它满足归一化条件,且处处不小于0. 假设初始条件下在$\bm{x}_0,\bm{p}_0$位置有个体积元$\mathrm{d}\bm{x}_0 \mathrm{d}\bm{p}_0$,跟踪这个体积元经历的轨线,达到$\mathrm{d}\bm{x}_t \mathrm{d}\bm{p}_t$时,在这个体积元的概率应为不变的。这可以理解为,根据Liouville定理,最开始在体积元里面的状态仍然会在初始状态演化后的体积元里。这可以表述为
    \begin{equation*}
        \rho(\bm{x}_t,\bm{p}_t) = \rho(\bm{x}_0,\bm{p}_0)
    \end{equation*}
    它对任意的$t$都成立,则
    \begin{equation*}
        \frac {\mathrm{d}\rho}{\mathrm{d}t} = \frac {\partial \rho}{\partial t} + \frac {\partial \rho}{\partial \bm{x}_t}\dot{\bm{x}}_t  + \frac {\partial \rho}{\partial \bm{p}_t} \dot{\bm{p}}_t = 0
    \end{equation*}
    再利用正则方程,得到
    \begin{equation*}
        - \frac {\partial \rho}{\partial t} = \bigg(\frac {\partial \rho}{\partial \bm{x}_t}\bigg)^\mathrm{T} \frac {\partial H}{\partial \bm{p}_t} - \bigg(\frac {\partial \rho}{\partial \bm{p}_t}\bigg)^\mathrm{T} \frac {\partial H}{\partial \bm{x}_t}
    \end{equation*}
    定义\textbf{Poisson括号}为
    \begin{equation*}
        \{ \rho, H\} = \bigg(\frac {\partial \rho}{\partial \bm{x}_t}\bigg)^\mathrm{T} \frac {\partial H}{\partial \bm{p}_t} - \bigg(\frac {\partial \rho}{\partial \bm{p}_t}\bigg)^\mathrm{T} \frac {\partial H}{\partial \bm{x}_t}
    \end{equation*}
    则有
    \begin{equation*}
        - \frac {\partial \rho}{\partial t} = \{ \rho, H\}
    \end{equation*}
    这也是Liouville定理的一种形式。如果Hamilton函数满足形式
    \begin{equation*}
        H(\bm{x}_t,\bm{p}_t) = \frac 12 \bm{p}_t^\mathrm{T} \bm{M}^{-1} \bm{p}_t + V(\bm{x}_t)
    \end{equation*}
    则有
    \begin{equation*}
        - \frac {\partial \rho}{\partial t} = \bigg(\frac {\partial \rho}{\partial \bm{x}_t}\bigg)^\mathrm{T} \bm{M}^{-1} \bm{p}_t  - \bigg(\frac {\partial \rho}{\partial \bm{p}_t}\bigg)^\mathrm{T} \frac {\partial V}{\partial \bm{x}_t}
    \end{equation*}

    一种常见的分布:\textbf{Boltzmann分布}:
    \begin{equation*}
        \rho(\bm{x},\bm{p}) \propto \mathrm{e}^{-\beta H(\bm{x},\bm{p})}
    \end{equation*}

    如果一个分布满足
    \begin{equation*}
        \frac {\partial \rho}{\partial t} = 0
    \end{equation*}
    则称为\textbf{稳态分布}。但是即使不是稳态分布,它也会满足对时间的全导数是0。这也是Liouville定理的一个形式。
    \begin{asg}
        第2次作业第1题: Boltzmann分布是否为稳态分布?
    \end{asg}

    研究一个概率密度的时候,有两种方式:一种是研究密度对时间的偏导,看静止空间的概率密度的变化,这称为\textbf{Euler图象}。另一种方式是研究密度对时间的劝导,跟踪状态运动的轨线,研究这个密度体积元在不同的时间的位置,这称为\textbf{Lagrange图象}。

    \bibliographystyle{plain}
    \bibliography{ref_chp_2}