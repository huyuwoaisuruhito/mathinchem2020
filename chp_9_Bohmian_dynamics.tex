\chapter{Bonmian 动力学}
    \section{连续性方程}

        类比经典情况, 若存在守恒量
        \begin{equation}
            \int \rho (\bm{x},t) \mathrm{d} \bm{x} = \text{const.}
        \end{equation}
        且局部没有粒子的产生与湮灭, 则有连续性方程
        \begin{equation}
            \frac {\partial \rho}{\partial t} + \bm{\nabla} \cdot (\rho \bm{v}) = 0
        \end{equation}
        可以将守恒量的方程两侧对时间求导, 可以得到一个在积分号下的连续性方程. 
        \begin{equation}\begin{aligned}
            \frac{\mathrm{d}}{\mathrm{d}t} \int_V \rho (\bm{x}_t, t) \mathrm{d} \bm{x}_t
            & = \int_V \bigg( \frac{\mathrm{d}\rho}{\mathrm{d}t} \mathrm{d} \bm{x}_t + \rho \frac{\mathrm{d}}{\mathrm{d}t} \mathrm{d} \bm{x}_t\bigg) \\
            & = \int_V \bigg( \frac{\partial \rho}{\partial t} + \frac{\partial \rho}{\partial \bm{x}_t}\bm{\dot{x}}_t + \rho \frac{\partial \bm{\dot{x}}_t}{\partial \bm{x}_t} \bigg) \mathrm{d} \bm{x}_t \\
            & = \int_V \bigg( \frac {\partial \rho}{\partial t} + \bm{\nabla} \cdot (\rho \bm{v}) \bigg) \mathrm{d} \bm{x}_t \\
            & = 0 
        \end{aligned}\end{equation}
        如果要求局部没有粒子的产生与湮灭, 该积分可以在任意小的体积元内成立, 则微分形式的连续性方程也成立. 

        在非相对论量子力学中粒子数守恒, 因此上述连续性方程依然成立. 
        \begin{equation}
            \int |\psi(\bm{x},t)|^2 \mathrm{d}x = 1
        \end{equation}

    \section{Bohmian 动力学}

        考虑含时Schrodinger方程
        \begin{equation}
            \mathrm{i}\hslash \frac {\partial}{\partial t} \psi(\bm{x},t) = \bigg( - \frac {\hslash^2}{2m}\bm{\nabla}^2 + V(\bm{x})\bigg) \psi(\bm{x},t)
        \end{equation}
        如果我们将波函数写成如下形式, 其中$\rho(\bm{x},t)$与$S(\bm{x},t)$是两个实函数, 且$\rho(\bm{x},t)$非负. 在后面我们会发现$\rho(\bm{x},t)$具有概率密度的意义, $S(\bm{x},t)$具有作用量的意义. 
        \begin{equation}
            \psi(\bm{x},t) = \sqrt{\rho(\bm{x},t)} \mathrm{e}^{\frac {\mathrm{i}S(\bm{x},t)t}{\hslash}}
        \end{equation}
        将波函数的形式代入含时Schrodinger方程. 对比方程两侧虚部与实部, 可以得到连续性方程以及Hamilton-Jacobian方程
        \begin{equation}\begin{aligned}
            &\frac {\partial \rho}{\partial t} + \bm{\nabla} \cdot (\rho \bm{v}) = 0 \\
            &\frac {1}{2m} \left(\frac {\partial S}{\partial x} \right)^2 + V(\bm{x}) - \frac {\hslash^2}{2m} \frac {\nabla^2 \sqrt{\rho}}{\sqrt{\rho}} = -\frac {\partial S}{\partial t}
        \end{aligned}\end{equation}
        其中连续性方程中的速度场为
        \begin{equation}
            \bm{v} = \frac 1m \frac {\partial S}{\partial \bm{x}}
        \end{equation}
        Hamilton-Jacobian方程前两项可以和经典情况类比, 第三项是由于量子效应产生的, 定义为\textbf{量子势}
        \begin{equation}
            Q(\bm{x},t) = - \frac {\hslash^2}{2m} \frac {\nabla^2 \sqrt{\rho}}{\sqrt{\rho}}
        \end{equation}

        从Hamilton-Jacobian方程出发, 可以得到运动方程
        \begin{equation}
            \frac 12 \bigg(\frac {\partial S}{\partial \bm{x}_t}\bigg)^{\mathrm{T}} \mb{M}^{-1} \frac {\partial S}{\partial \bm{x}_t} + V(\bm{x}) + Q(\bm{x}, t) = -\frac {\partial S}{\partial t}
        \end{equation}
        考虑作用量的全导
        \begin{equation}
            \frac {\mathrm{d}S}{\mathrm{d}t} = \frac {\partial S}{\partial t} + \frac {\partial S}{\partial \bm{x}_t} \dot{\bm{x}_t} = \frac 12 \bigg(\frac {\partial S}{\partial \bm{x}_t}\bigg)^{\mathrm{T}} \mb{M}^{-1} \frac {\partial S}{\partial \bm{x}_t} - V(\bm{x}) - Q(\bm{x}, t)
        \end{equation}
        之后可以对位置求偏导(即对方程两侧求${\partial} / {\partial \bm{x}_t}$), 最终总结得到以下第二个式子. 配以动量的定义, 可以得到$\bm{x}$与$\bm{p}$的演化方程. 
        \begin{equation}\begin{aligned}
            \dot{\bm{x}}_t &= \mb{M}^{-1} \frac {\partial S(\bm{x}_t,t)}{\partial \bm{x}_t} \\
            \dot{\bm{p}}_t &= -\frac {\partial V(\bm{x}_t)}{\partial \bm{x}_t} - \frac {\partial Q(\bm{x}_t, t)}{\partial \bm{x}_t}
        \end{aligned}\end{equation}
        这称为\textbf{量子轨线方程}. 如果对$\rho$求全导
        \begin{equation}
            \frac {\mathrm{d}\rho}{\mathrm{d}t} = \frac {\partial \rho}{\partial t} + \frac {\partial \rho}{\partial \bm{x}_t} \dot{\bm{x}}_t = -\rho \bm{\nabla} \cdot \dot{\bm{x}}_t
        \end{equation}
        由以上方程, 可以通过直接演化量子轨线计算量子体系随时间演化的问题. 但计算速度场的散度在数值上无疑是十分困难的. 

        \bibliographystyle{plain}
        \bibliography{ref_chp_9}