\chapter{Liouville 定理}
    \section{重新审视Hamilton方程}
    \subsection{将Hamilton方程写为对称的形式}
    由于Hamilton方程是一阶微分方程组,其变量$\bm{x},\bm{p}$拥有比牛顿方程中位置和速度更加平等的地位
    \footnote{事实上,正则变量之间平等地位更应该通过正则变换\cite{Goldstein2000Classical}来说明},
    所以我们希望能将Hamilton方程写为对称的形式。对于一个$n$自由度的力学系统
    \footnote{
        即$\bm{x}$是一个n维向量
    }
    ,考虑引入一个新变量$\bm{\eta}:= (\bm{x}, \bm{p})$
    ,那么正则方程可以写为:
    \begin{equation}
        \left\{
        \begin{split}
            \dot{\eta}_{i} &= \dot{x}_i = \pdv{H}{p_i} = \pdv{H}{\eta_{i+n}}\\
            \dot{\eta_{i+n}} &= \dot{p}_i = - \pdv{H}{x_i} = -\pdv{H}{\eta_{i}}
        \end{split}
        \right.
        \quad\quad\quad i = 1,\cdots, n
    \end{equation}
    引入$2n\times 2n$方阵$\bm{J}$:
    \begin{equation}
        \bm{J} = 
        \begin{bmatrix}
            \bm{0} & \bm{I}_n\\
            -\bm{I}_n & \bm{0}
        \end{bmatrix}
    \end{equation}
    那么Hamilton方程可以写为:
    \begin{equation}
        \dot{\bm{\eta}} = \bm{J}\pdv{H}{\bm{\eta}}
    \end{equation}
    \subsection{动力系统的概念}
    这一节讨论一下比Hamilton方程更加普遍的情况。考虑如下微分方程组(Hamilton方程显然是下面所述
    微分方程的一个特例):
    \begin{equation}
        \dv{\bm{x}}{t} = \bm{v}(\bm{x})
    \end{equation}
    其中$\bm{v}$是一个$\mathbb{R}^{n}\to\mathbb{R}^{n}$的连续可微映射。这个方程可以看作
    一个在给定速度场中运动的粒子的运动方程。对于任何给定的初始条件:
    \begin{equation}
        \bm{x}(t_0) = \bm{x_0}
    \end{equation}
    方程满足微分方程解的存在唯一性定理\cite{丁同仁2004常微分方程教程},因此对于初始条件有唯一解:
    \begin{equation}
        \bm{x} = \bm{\phi}(t;t_0, \bm{x}_0)
    \end{equation}
    我们将$\bm{\phi}$在给定$t_0,\bm{x}_0$时作为$t$的函数的“函数图像”
    \footnote{即集合$\{(\bm{\phi}(t;t_0,\bm{x}_0), t)|t>t_0\}$}称为积分曲线。
    其中$\bm{x}$取值的空间$\mathbb{R}^n$称为\textbf{相空间}。给定初始条件之后微分方程的解
    给出了相空间中一条以$t$为参数并且在$t_0$时间通过$\bm{x}_0$的曲线,称之为\textbf{轨线}。
    从直观上讲,这个曲线就是我们能看到的在速度场中运动的粒子所走过的轨迹。一般称这样的方程为动力系统
    (注意速度场并不显含时间)。
    \par 下面介绍几个动力系统的基本性质,有助于我们理解本节讲授的Liouville定理。
    \par (1) 积分曲线的平移不变性。考虑$\bm{x} = \bm{\phi}(t;t_0,\bm{x}_0)$是初始条件(2.5)
    对应的解,那么$\bm{x} = \bm{\phi}(t;t_0 + C,\bm{x}_0)$依然满足方程(但是不再满足初始条件),
    这个结论的正确性是由速度场不含时间保证的,可以直接将解带入微分方程进行验证。容易想象,时间平移之后
    相空间中的轨线完全没有变化,可以形象地理解,在不含时速度场中粒子运动的轨迹只取决于粒子的初始位置
    而与粒子开始运动的时间无关。
    \par (2) 过相空间每一点轨线的唯一性。 这是动力系统(同样是Hamilton方程)的一个重要性质,
    它说明了从相空间中不同点出发的轨线不可能相交。考虑两个不同初始条件的积分曲线
    $\bm{\phi}(t;t_1,\bm{x}_1),\, \bm{\phi}(t; t_2, \bm{x}_2)$在相空间中相交于同一点$\bm{x}'$,
    那么可以对其中一个曲线进行时间平移,使得在某个时刻$t'$两个积分曲线相交于$(\bm{x}', t')$,由微分
    方程解的存在唯一性定理,这两个积分曲线必须完全重合,那么它们对应的相空间中的轨线也必须完全重合,
    这说明了通过相空间中每一点有且只有一条轨线。
    \par (3) 相流。考虑将初始条件设为$t=0,\, \bm{x}(0) = \bm{x}_0$,定义
    $\mathbb{R}^n \to \mathbb{R}^n$的映射:
    \begin{equation}
        \phi^{t}: \bm{x}_{0} \to \bm{\phi}(t; 0, \bm{x}_0)
    \end{equation}
    这个映射将$t=0$时粒子在相空间中的位置映射为粒子沿着轨线运动$t$时间后粒子在相空间中的位置。由轨线
    的唯一性可知,对于$\forall t$映射$\phi^t$是一个双射。更进一步,由于已经假设了$\bm{v}(\bm{x})$
    是一个连续可微映射,因此$\bm{\phi}(t; 0, \bm{x}_0)$对于初值$\bm{x}_0$也是连续可微的
    \cite{丁同仁2004常微分方程教程2},那么映射$\phi^t$是$\mathbb{R}^n$上的一个微分同胚。事实上,
    容易根据微分方程解的存在唯一性证明$\forall s, t\in \mathbb{R}$:
    \begin{equation}
        \phi^{s + t} = \phi^s\circ\phi^{t}
    \end{equation}
    这样$\phi^t$组成的集合就有了群的结构
    \footnote{严格来讲,应该验证存在单位元、逆元等,使这个集合满足群的定义}
    ,将这样的一个单参数微分同胚群称为\textbf{相流}。

    \subsection{Hamilton系统的稳定性矩阵}
    在这一节中我们讨论改变初值对于轨线的影响。固定初始时刻$t=0$,初始位置为$(\bm{x}_0, \bm{p}_0)$,
    将$t$时刻系统在相空间中的位置写成初始条件的函数(这个函数就是Hamilton方程的解):
    \begin{equation}
        \begin{split}
            \bm{x}_t &= \bm{x}_t(\bm{x}_0, \bm{p}_0)\\
            \bm{p}_t &= \bm{p}_t(\bm{x}_0, \bm{p}_0)
        \end{split}
    \end{equation}
    若初始位置偏离$(\mathrm{d}\bm{x}_0,\mathrm{d}\bm{p}_0)$(假设偏离量是小量),那么
    \begin{equation}
        \begin{split}
        \bm{x}_t(\bm{x}_0+\mathrm{d}\bm{x}_0, \bm{p}_0+\mathrm{d}\bm{p}_0) &= \bm{x}_t(\bm{x}_0, \bm{p}_0) + \frac {\partial \bm{x}_t}{\partial \bm{x_0}} \d \bm{x}_0 + \frac {\partial \bm{x}_t}{\partial \bm{p}_0} \d \bm{p}_0\\
        \bm{p}_t(\bm{x}_0+\mathrm{d}\bm{x}_0, \bm{p}_0+\mathrm{d}\bm{p}_0) &= \bm{p}_t(\bm{x}_0, \bm{p}_0) + \frac {\partial \bm{p}_t}{\partial \bm{x_0}} \d \bm{x}_0 + \frac {\partial \bm{p}_t}{\partial \bm{p}_0} \d \bm{p}_0
        \end{split}
    \end{equation}
    这里只考虑了Taylor展开到一阶的结果。或者写成
    \begin{equation}
        \begin{split}
            \mathrm{d}\bm{x}_t &= \frac {\partial \bm{x}_t}{\partial \bm{x_0}} \d \bm{x}_0 + \frac {\partial \bm{x}_t}{\partial \bm{p}_0} \d \bm{p}_0\\
            \mathrm{d}\bm{p}_t &= \frac {\partial \bm{p}_t}{\partial \bm{x_0}} \d \bm{x}_0 + \frac {\partial \bm{p}_t}{\partial \bm{p}_0} \d \bm{p}_0
        \end{split}
        \end{equation}
    现在已经将$t$时刻的位置偏离表示成了0时刻位置偏离的函数\footnote{初始时刻只有无穷小偏离,因此只考虑线性近似}
    ,但是由于Hamilton方程的解是未知的,系数矩阵没有办法直接求出来。为了获取系数矩阵的一些信息,我们尝试对时间求导:
    \begin{equation}
        \begin{split}
            \frac{\mathrm{d}}{\mathrm{d}t} \bigg(\frac {\partial \bm{x}_t}{\partial \bm{x}_0}\bigg)_{\bm{p}_0} &= \bigg(\frac {\partial}{\partial \bm{x}_0} \frac {\mathrm{d}\bm{x}_t}{\mathrm{d}t}\bigg)_{\bm{p}_0} = \bigg(\frac {\partial}{\partial \bm{x}_0} \bigg(\frac {\partial H}{\partial \bm{p}_t}\bigg)_{\bm{x}_t}\bigg)_{\bm{p}_0} = \bigg(\frac {\partial^2 H}{\partial \bm{x}_t \partial \bm{p}_t}\bigg) \bigg(\frac {\partial \bm{x}_t}{\partial \bm{x}_0}\bigg)_{\bm{p}_0}+ \bigg(\frac {\partial^2H}{\partial \bm{p}_t^2}\bigg)_{\bm{x}_t} \bigg(\frac {\partial \bm{p}_t}{\partial \bm{x}_0}\bigg)_{\bm{p}_0}\\
            \frac{\mathrm{d}}{\mathrm{d}t} \bigg(\frac {\partial \bm{x}_t}{\partial \bm{p}_0}\bigg)_{\bm{x}_0} &= \bigg(\frac {\partial}{\partial \bm{p}_0} \frac {\mathrm{d}\bm{x}_t}{\mathrm{d}t}\bigg)_{\bm{x}_0} = \bigg(\frac {\partial}{\partial \bm{p}_0} \bigg(\frac {\partial H}{\partial \bm{p}_t}\bigg)_{\bm{x}_t}\bigg)_{\bm{x}_0} = \bigg(\frac {\partial^2 H}{\partial \bm{x}_t \partial \bm{p}_t}\bigg) \bigg(\frac {\partial \bm{x}_t}{\partial \bm{p}_0}\bigg)_{\bm{x}_0} + \bigg(\frac {\partial^2H}{\partial \bm{p}_t^2}\bigg)_{\bm{x}_t} \bigg(\frac {\partial \bm{p}_t}{\partial \bm{p}_0}\bigg)_{\bm{x}_0}\\
            \frac{\mathrm{d}}{\mathrm{d}t} \bigg(\frac {\partial \bm{p}_t}{\partial \bm{x}_0}\bigg)_{\bm{p}_0} &= \bigg(\frac {\partial}{\partial \bm{x}_0} \frac {\mathrm{d}\bm{p}_t}{\mathrm{d}t}\bigg)_{\bm{p}_0} = -\bigg(\frac {\partial}{\partial \bm{x}_0} \bigg(\frac {\partial H}{\partial \bm{x}_t}\bigg)_{\bm{p}_t}\bigg)_{\bm{p}_0} = -\bigg(\frac {\partial^2 H}{\partial \bm{x}_t^2}\bigg)_{\bm{p}t} \bigg(\frac {\partial \bm{x}_t}{\partial \bm{x}_0}\bigg)_{\bm{p}_0}- \bigg(\frac {\partial^2H}{\partial \bm{p}_t \partial \bm{x}_t}\bigg) \bigg(\frac {\partial \bm{p}_t}{\partial \bm{x}_0}\bigg)_{\bm{p}_0}\\
            \frac{\mathrm{d}}{\mathrm{d}t} \bigg(\frac {\partial \bm{p}_t}{\partial \bm{p}_0}\bigg)_{\bm{x}_0} &= \bigg(\frac {\partial}{\partial \bm{p}_0} \frac {\mathrm{d}\bm{p}_t}{\mathrm{d}t}\bigg)_{\bm{x}_0} = -\bigg(\frac {\partial}{\partial \bm{p}_0} \bigg(\frac {\partial H}{\partial \bm{x}_t}\bigg)_{\bm{p}_t}\bigg)_{\bm{x}_0} = -\bigg(\frac {\partial^2 H}{\partial \bm{x}_t^2}\bigg)_{\bm{p}t} \bigg(\frac {\partial \bm{x}_t}{\partial \bm{p}_0}\bigg)_{\bm{x}_0}- \bigg(\frac {\partial^2H}{\partial \bm{p}_t \partial \bm{x}_t}\bigg) \bigg(\frac {\partial \bm{p}_t}{\partial \bm{p}_0}\bigg)_{\bm{x}_0}    
        \end{split}
    \end{equation}
    由此可以得到一个微分方程组:
    \begin{equation}
        \displaystyle
        \begin{split}
            \frac {\mathrm{d}}{\mathrm{d}t}
            \begin{bmatrix}
                \frac {\partial \bm{x}_t}{\partial \bm{x}_0} & \frac {\partial \bm{x}_t}{\partial \bm{p}_0}\\
                \frac {\partial \bm{p}_t}{\partial \bm{x}_0} & \frac {\partial \bm{p}_t}{\partial \bm{p}_0}
            \end{bmatrix}
            =
            \begin{bmatrix}
                \frac {\partial^2 H}{\partial \bm{x}_t \partial \bm{p}_t} & (\frac {\partial^2 H}{\partial \bm{p}_t^2})_{\bm{x}_t}\\
                -(\frac {\partial^2 H}{\partial \bm{x}_t^2})_{\bm{p}_t} & - \frac {\partial^2 H}{\partial \bm{x}_t \partial \bm{p}_t}
            \end{bmatrix}
            \begin{bmatrix}
                \frac {\partial \bm{x}_t}{\partial \bm{x}_0} & \frac {\partial \bm{x}_t}{\partial \bm{p}_0}\\
                \frac {\partial \bm{p}_t}{\partial \bm{x}_0} & \frac {\partial \bm{p}_t}{\partial \bm{p}_0}
            \end{bmatrix}
        \end{split}
    \end{equation}
    等式右边由Hamilton量二阶偏导数组成的矩阵称为Hamilton系统的\textbf{稳定性矩阵},其决定了在相空间
    某一点附近,初值改变引起的\textbf{短时间内}轨线改变的趋势,下面给以简单说明。考虑一般的微分方程组
    (2.4),初始条件分别为$t=0, \bm{x}(0) = \bm{x}_0;\, t=0, \bm{x}(0) = \bm{x}_0 + \delta\bm{x}_0$
    ,写出对应的解:
    \begin{equation}
        \begin{split}
            \dot{\bm{\phi}}(t, \bm{x}_0) &= \bm{v}(\bm{\phi}(t, \bm{x}_0))\\
            \dot{\bm{\phi}}(t, \bm{x}_0 + \delta\bm{x}_0) &= \bm{v}(\bm{\phi}(t, \bm{x}_0 + \delta\bm{x}_0))
        \end{split}
    \end{equation}
    将上两式相减,得到改变初值引起的积分曲线的变化所满足的方程:
    \begin{equation}
        \dot{\delta\bm{\phi}} = \left.\pdv{\bm{v}}{\bm{x}}\right|_{\bm{x = \bm{\phi(t, \bm{x}_0)}}} \delta\bm{\phi}
    \end{equation}
    上式是一个线性的微分方程,其系数矩阵显然是时间的函数,如果我们认为在较短的时间内系数矩阵不随时间变化,
    而且取初始位置$\bm{x}_0$的值,那么上述方程在短时间内的解近似为:
    \begin{equation}
        \delta \bm{\phi} \approx \exp\left(\left.\pdv{\bm{v}}{\bm{x}}\right|_{\bm{x} = \bm{x}_0}\right) \delta\bm{x}_0
    \end{equation}
    可以看到短时间内矩阵$\displaystyle\left.\pdv{\bm{v}}{\bm{x}}\right|_{\bm{x} = \bm{x}_0}$的性质决定了$\delta\bm{\phi}$
    的行为(主要是其特征值实部的正负\footnote{若实部最大的特征值实部若大于0,则在该点邻域内初值小的偏移
    都会引起轨线以指数增长的偏移,称这样的区域为轨道的不稳定区域;若特征值实部全部小于0,那么初值小的偏移
    引起的轨线偏移是指数衰减的,称这样的区域为轨道的稳定区域})。对于Hamilton系统,$\displaystyle\bm{v} = \bm{J}\pdv{H}{\bm{\eta}}$,
    容易验证$\displaystyle\pdv{\bm{v}}{\bm{x}}$就是之前定义的Hamilton系统的稳定性矩阵。
    \section{Liouville定理}
    \subsection{Liouville定理的提出}
    前一节已经指出了$\phi^t$是一个$\mathbb{R}^n\to\mathbb{R}^n$的微分同胚,其意义是将0时刻相空间
    中的点映射到$t$时刻相空间中的点。我们关心这个映射的性质,对于Hamilton系统(2.3)
    一个重要的性质是$\phi^t$是一个保持体积的映射,具体来说,假设$V(D)$表示
    $D\in\mathbb{R}^n$(前提是$D$的体积有定义)的体积,那么:
    \begin{equation}
        V(\phi^t(D)) = V(D)
    \end{equation}
    这个结论并不需要用到Hamilton系统的全部性质,可以期待$\phi^t$保持了相空间中更多的量。同时也可以探究
    一般的动力系统(2.4)中相体积在$\phi^t$下的变化。
    \subsection{Liouville定理的证明}
    $t$时刻$\phi^t(D)$的体积用积分表示为:
    \begin{equation}
        V(\phi^t(D)) = \int_{\phi^t D}\d x_1\d x_2\cdots\d x_n
    \end{equation}
    考虑到重积分的换元公式,可以将$\phi^t(D)$的体积写为:
    \begin{equation}
        V(\phi^t(D)) = \int_{D}\left|\pdv{\phi^t}{\bm{x}}\right|\d x_1\d x_2\cdots\d x_n
    \end{equation}
    那么只用知道映射$\phi^t$的Jacobi行列式$\displaystyle\left|\pdv{\phi^t}{\bm{x}}\right|$的表达式就可以计算出
    $\phi^t$映射下体积的变化。但是这个行列式本身的性质无法直接看出,我们对其求导:
    \begin{equation}
        \begin{split}
            \dv{}{t}\pdv{\phi^t}{\bm{x}_0} &= \pdv{}{\bm{x}_0}\dv{\bm{\phi}(t, \bm{x}_0)}{t}\\
            & = \pdv{\bm{v}(\bm{\phi}(t, \bm{x}_0))}{\bm{x}_0}\\
            & = \left.\pdv{\bm{v}(\bm{x})}{\bm{x}}\right|_{\bm{x}=\bm{\phi}(t, \bm{x}_0)}\cdot
            \pdv{\phi^t}{\bm{x}_0}
        \end{split}
    \end{equation}
    我们希望通过上面得到的等式来计算Jacobi行列式$\displaystyle\pdv{\phi^t}{\bm{x}_0}$。
    \par 
    首先给出一个利用矩阵函数性质的证明。考虑方阵$\bm{A}$:
    \begin{equation}
        \bm{A} = 
        \begin{pmatrix}
            a_{11} & \cdots & a_{1n}\\
            \vdots & & \vdots\\
            a_{n1} & \cdots & a_{nn}\\
        \end{pmatrix}
    \end{equation}
    它的行列式可以表示为(行列式按行展开):
    \begin{equation}
        \det{\bm{A}} = \sum_{j=1}^na_{ij} A_{ij}^* \quad\quad i = 1, 2, \cdots, n
    \end{equation}
    其中,$A_{ij}^*$表示$a_{ij}$的代数余子式
    \footnote{方阵$\bm{A}\,$$i,j$元$a_{ij}$余子式$M_{ij}$定义为将方阵$\bm{A}$的i行j列去掉后组成的方阵的行列式,
    容易通过行列式的多重线性性和交错对称性证明:
    \begin{equation}
        \det \bm{A} = \sum_{j = 1}^{n}(-1)^{i + j}a_{ij}\cdot M_{ij}\quad \quad i = 1, 2, \cdots, n
    \end{equation}
    $a_{ij}$的代数余子式$A_{ij}^*$定义为$A_{ij}^* := (-1)^{i+j}M_{ij}$
    }
    。定义$\bm{A}$的\textbf{伴随矩阵}$\bar{\bm{A}}$为:
    \begin{equation}
        \bar{\bm{A}}_{ij} = A_{ji}^* 
    \end{equation}
    矩阵$\bm{A}$的逆矩阵(假设可逆)可以用伴随矩阵$\bar{\bm{A}}$表示为
    \footnote{可以通过矩阵乘法与行列式的性质验证下面所表示的矩阵就是$\bm{A}$的逆矩阵}:
    \begin{equation}
        \bm{A}^{-1} = \frac{1}{\det{\bm{A}}}\bm{\bar{A}}
    \end{equation}
    \par 
    假设方阵$\bm{A}$中的每个元素$a_{ij}$均是$t$的函数,可以想象对于方阵$\bm{A}$求导即是对于每个
    元素求导组成的方阵,但是对$\det \bm{A}$求导并不是这样。利用行列式是每一行(每一列)的多重线性
    函数这个性质
    \footnote{
        设$T(v_1,\cdots,v_n)$是$V\times\cdots\times V \to \mathbb{R}$的多重线性映射,那么:
        \begin{equation}
            \dv{T(v_1,\cdots,v_n)}{t} = T\left(\dv{v_1}{t}, v_2, \cdots, v_n\right) + T\left(v_1, \dv{v_2}{t}, \cdots, v_n\right)
             + \cdots + T\left(v_1, v_2, \cdots, \dv{v_n}{t}\right)
        \end{equation}
        有很多与之相关的例子,比如对n个函数乘积的求导(莱布尼茨法则),对两个向量内积的求导、
        对$\mathbb{R}^3$中向量积的求导、对对易子的求导等等;证明的思路与莱布尼茨法则证明的思路类似。
    },可以给出对行列式求导的结果:
    \begin{equation}
        \frac {\mathrm{d}}{\mathrm{d}t} \det{\bm{A}} = \sum_i \det{\dot{\bm{A}_{i}}}
    \end{equation}
    其中,$\dot{\bm{A}_{i}}$是只对第$i$行的所有元素对时间求导,其他元素不变得到的矩阵。进一步得到
    \begin{equation}
        \begin{split}
            \frac {\mathrm{d}}{\mathrm{d}t} \det{\bm{A}} &= \sum_i \det{\dot{\bm{A}_i}} = \sum_{i=1}^n \sum_{j=1}^n \frac {\mathrm{d}a_{ij}}{\mathrm{d}t} \bm{A}_{ij}^*\\
        &= \mathrm{Tr} \bigg(\frac {\mathrm{d}\bm{A}}{\mathrm{d}t} \bar{\bm{A}}\bigg) = \mathrm{Tr} \bigg(\frac {\mathrm{d}\bm{A}}{\mathrm{d}t} \bm{A}^{-1}\bigg) \det{\bm{A}}
        \end{split}
    \end{equation}
    将两边同时除以$\bm{A}$的行列式,得到一个重要的公式:
    \begin{equation}
        \frac {\mathrm{d}}{\mathrm{d}t} \ln{\det{\bm{A}}} = \mathrm{Tr} \bigg(\frac {\mathrm{d}\bm{A}}{\mathrm{d}t} \bm{A}^{-1}\bigg)
    \end{equation}
    假设$\bm{A}$满足(就是$\phi^t$的Jacobi行列式满足的条件):
    \begin{equation}
        \frac {\mathrm{d}}{\mathrm{d}t} \bm{A} = \bm{M}\cdot\bm{A}
    \end{equation}
    就有:
    \begin{equation}
        \frac {\mathrm{d}}{\mathrm{d}t} \ln{\det{\bm{A}}} = \mathrm{Tr}\ \bm{M}
    \end{equation}
    将这个结论应用于$\phi^t$满足的方程(2.20),可以得到:
    \begin{equation}
        \dv{}{t}\ln \left|\pdv{\phi^t}{\bm{x}_0}\right| = \mr{Tr}\left(\pdv{\bm{v}}{\bm{x}}\right)_{\bm{x} = \bm{\phi}(t, \bm{x}_0)}
    \end{equation}
    将上式两边对时间积分,同时考虑到初始条件$\displaystyle\pdv{\phi^0}{\bm{x}_0} = 1$,可以给出$t$时刻Jacobi矩阵
    的表达式:
    \begin{equation}
        \left|\pdv{\phi^t}{\bm{x}_{0}}\right| = \exp\left(\int_{0}^{t}\mr{Tr}\left.
        \pdv{\bm{v}}{\bm{x}}\right|_{\bm{x} = \bm{\phi}(u, \bm{x}_0)}\d u\right)
    \end{equation}
    将上式带入(2.19)就可以得到理论上的体积公式。可以看出,要计算出Jacobi行列式必须要方程的显式解,
    这在大多数情况下并不能够得到满足,但是当$\displaystyle\mr{Tr}\pdv{\bm{v}}{\bm{x}}$
    与时间无关时,就容易计算出Jacobi行列式。有一种特殊的情况:
    \begin{equation}
        \mr{Tr}\pdv{\bm{v}}{\bm{x}} = 0
    \end{equation}
    此时$\displaystyle\left|\pdv{\phi^t}{\bm{x}_0}\right| = 1,\, \forall t$,
    满足这个条件的系统的相流中的任何一个元素$\phi^t$均是保持相体积不变的映射,
    这个结论被称为Liouville定理。对于Hamilton系统的相流中某个映射$\phi^t$的Jacobi行列式有
    (参考(2.30)中的定义):
    \begin{equation}
        \bm{M} = \bm{J} \pdv{^2 H}{\bm{\eta}^2}=
        \begin{pmatrix}
        \frac {\partial^2 H}{\partial \vec{x}_t \partial \vec{p}_t} & (\frac {\partial^2 H}{\partial \vec{p}_t^2})_{\vec{x}_t}\\
        -(\frac {\partial^2 H}{\partial \vec{x}_t^2})_{\vec{p}_t} & - \frac {\partial^2 H}{\partial \vec{x}_t \partial \vec{p}_t}
        \end{pmatrix}
    \end{equation}
    显然这个矩阵的迹为0,所以
    \begin{equation}
        \frac {\mathrm{d}}{\mathrm{d}t} \det \bigg|\frac {\partial (\bm{x}_t,\bm{p}_t)}{\partial (\bm{x}_0,\bm{p}_0)} \bigg| = 0
    \end{equation}
    所以Jacobi行列式恒为1。由之前得到的结果(2.19)就可以证明Hamilton相流中任一映射均是保持相体积的映射。
    为了区分0时刻与$t$时刻的相空间,通常将0时刻相空间中的坐标记为$(\bm{x}_0, \bm{p}_0)$,$t$时刻记为
    $(\bm{x}_t,\bm{p}_{t})$\footnote{这样的记法所表达的真正含义是$\phi^t(\bm{x}_0, \bm{p}_0) = (\bm{x}_t, \bm{p}_t)$},
    那么两个由映射$\phi^t$联系起来的空间的体积元之间满足:
    \begin{equation}
        \mathrm{d}\bm{x}_t \mathrm{d}\bm{p}_t = \left|\pdv{\phi^t}{\bm{x}_0}\right|\mathrm{d}\bm{x}_0 \mathrm{d}\bm{p}_0 = \d \bm{x}_0\d \bm{p}_0
    \end{equation}
    上面的等式完全可以看作Liouville定理的另一种表述形式。
    \par
    还有另外的证明Liouville定理的方法
    
    \section{Homework}
    \begin{asg}
        尝试用不同的方法证明Liouville定理。
    \end{asg}


    \bibliographystyle{plain}
    \bibliography{ref_chp_2}