\chapter{Bonmian 动力学}
    \section{20210104:Bohmian动力学}

        类比经典情况,若存在守恒量
        \begin{equation*}
            \int \rho (\bm{x},t) \mathrm{d} \bm{x} = \text{const.}
        \end{equation*}
        且局部没有粒子的产生与湮灭,则有连续性方程
        \begin{equation*}
            \frac {\partial \rho}{\partial t} + \bm{\nabla} \cdot (\rho \bm{v}) = 0
        \end{equation*}
        可以将守恒量的方程两侧对时间求导,可以得到一个在积分中的连续性方程
        \begin{align*}
            \frac{\mathrm{d}}{\mathrm{d}t} \int_V \rho (\bm{x}_t, t) \mathrm{d} \bm{x}_t
            & = \int_V \bigg( \frac{\mathrm{d}\rho}{\mathrm{d}t} \mathrm{d} \bm{x}_t + \rho \frac{\mathrm{d}}{\mathrm{d}t} \mathrm{d} \bm{x}_t\bigg) \\
            & = \int_V \bigg( \frac{\partial \rho}{\partial t} + \frac{\partial \rho}{\partial \bm{x}_t}\bm{\dot{x}}_t + \rho \frac{\partial \bm{\dot{x}}_t}{\partial \bm{x}_t} \bigg) \mathrm{d} \bm{x}_t \\
            & = \int_V \bigg( \frac {\partial \rho}{\partial t} + \bm{\nabla} \cdot (\rho \bm{v}) \bigg) \mathrm{d} \bm{x}_t \\
            & = 0 
        \end{align*}
        如果希望连续性方程在局部也成立,则要求局部没有粒子的产生与湮灭。

        在量子力学中也有守恒量
        \begin{equation*}
            \int |\psi(\bm{x},t)|^2 \mathrm{d}x = 1
        \end{equation*}
        因此上述连续性方程依然成立。

        考虑含时Schrodinger方程
        \begin{equation*}
            \mathrm{i}\hbar \frac {\partial}{\partial t} \psi(\bm{x},t) = \bigg( - \frac {\hbar^2}{2m}\bm{\nabla}^2 + V(\bm{x})\bigg) \psi(\bm{x},t)
        \end{equation*}
        如果我们将波函数写成如下形式,其中$\rho(\bm{x},t)$与$S(\bm{x},t)$是两个实函数,且$\rho(\bm{x},t)$非负。在后面我们会发现$\rho(\bm{x},t)$具有概率密度的意义,$S(\bm{x},t)$具有作用量的意义。
        \begin{equation*}
            \psi(\bm{x},t) = \sqrt{\rho} \mathrm{e}^{\frac {\mathrm{i}S(\bm{x},t)t}{\hbar}}
        \end{equation*}
        将波函数的形式代入含时Schrodinger方程,对比方程两侧虚部与实部,可以得到连续性方程以及Hamilton-Jacobian方程
        \begin{align*}
            &\frac {\partial \rho}{\partial t} + \bm{\nabla} \cdot (\rho \bm{v}) = 0 \\
            &\frac {(\frac {\partial S}{\partial x})^2}{2m} + V(\bm{x}) - \frac {\hbar^2}{2m} \frac {\nabla^2 \sqrt{\rho}}{\sqrt{\rho}} = -\frac {\partial S}{\partial t}
        \end{align*}
        其中连续性方程中的速度场为
        \begin{equation*}
            \bm{v} = \frac 1m \frac {\partial S}{\partial \bm{x}}
        \end{equation*}
        Hamilton-Jacobian方程前两项可以和经典情况类比,第三项是量子力学而来,定义为\textbf{量子势}:
        \begin{equation*}
            Q(\bm{x},t) = - \frac {\hbar^2}{2m} \frac {\nabla^2 \sqrt{\rho}}{\sqrt{\rho}}
        \end{equation*}

        从Hamilton-Jacobian方程出发,可以得到运动方程
        \begin{equation*}
            \frac 12 \bigg(\frac {\partial S}{\partial \bm{x}_t}\bigg)^{\mathrm{T}} \bm{M}^{-1} \frac {\partial S}{\partial \bm{x}_t} + V(\bm{x}) + Q(\bm{x}, t) = -\frac {\partial S}{\partial t}
        \end{equation*}
        考虑作用量的全导
        \begin{equation*}
            \frac {\mathrm{d}S}{\mathrm{d}t} = \frac {\partial S}{\partial t} + \frac {\partial S}{\partial \bm{x}_t} \dot{\bm{x}_t} = \frac 12 \bigg(\frac {\partial S}{\partial \bm{x}_t}\bigg)^{\mathrm{T}} \bm{M}^{-1} \frac {\partial S}{\partial \bm{x}_t} - V(\bm{x}) - Q(\bm{x}, t)
        \end{equation*}
        之后可以对位置求偏导(即对方程两侧求${\partial} / {\partial \bm{x}_t}$),最终得到以下第二个式子。第一个式子是动量的定义。
        \begin{align*}
            \dot{\bm{x}}_t &= \bm{M}^{-1} \frac {\partial S(\bm{x}_t,t)}{\partial \bm{x}_t} \\
            \dot{\bm{p}}_t &= -\frac {\partial V(\bm{x}_t)}{\partial \bm{x}_t} - \frac {\partial Q(\bm{x}_t, t)}{\partial \bm{x}_t}
        \end{align*}
        这称为\textbf{量子轨线方程}。如果对$\rho$求全导,
        \begin{equation*}
            \frac {\mathrm{d}\rho}{\mathrm{d}t} = \frac {\partial \rho}{\partial t} + \frac {\partial \rho}{\partial \bm{x}_t} \dot{\bm{x}}_t = -\rho \bm{\nabla} \cdot \dot{\bm{x}}_t
        \end{equation*}
        由以上方程,可以通过直接演化量子轨线计算量子体系随时间演化的问题。但计算速度场的散度在数值上无疑是十分困难的。
        \bibliographystyle{plain}
        \bibliography{ref_chp_9}