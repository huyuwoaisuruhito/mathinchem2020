\chapter{Lagrange力学和量子力学的路径积分形式}
    \section{作用量与Lagrange力学}

        在前一章我们已经讨论了很多关于基于哈密顿力学的量子力学, 但这不是量子力学唯一的表示形式. 在经典力学中我们知道要描述力学体系, 除了使用基于哈密顿量的Hamilton力学, 还可以使用Lagrange力学作为一种完全等效的描述手段. 在探索量子力学的其他的表示方法之前, 我们需要先回顾一下经典力学中除Hamilton力学之外其他的两种描述力学系统的手段: 通过Lagrange力学, 或者通过Hamilton-Jacobi方程. 

        \subsection{最小作用量原理}

        在本讲义的最开始就已经给出了Hamilton力学中的Hamilton正则方程, 并且省略了推导过程, 这里我们就来填这个坑. 

        首先我们需要给出描述经典力学中最为重要的力学原理, \textbf{最小作用量原理}. 它完全概括了经典力学体系的运动规律. 下面这个定理摘自刘川老师的理论力学讲义(version1.4). 

        \begin{law}[最小作用量原理]
            力学体系有一个与其运动相关的物理量称为\textbf{作用量}$S$, 它是一个洛伦兹标量. 如果一个力学体系在给定时刻$t_1$和$t_2$分别由给定的广义坐标$q^{(1)}$与$q^{(2)}$描写, 则该力学系统的作用量$S$可以表达为联结这两个位型之间各种可能轨迹的\textbf{泛函}. 
            \begin{equation}
                S[\bm q] = \int_{t_1}^{t_2} L( \bm q(t), \bm \dot q(t), t ) \dd t
            \end{equation}
            这里的函数$L(\bm q,\bm \dot q,t)$称为系统的\textbf{拉格朗日量}, 该力学体系在时刻$t_1$和$t_2$之间联结广义坐标$\bm q^{(1)}$与$\bm q^{(2)}$的真实运动轨迹就是使作用量$S$的一阶变分$\delta S = 0$的轨迹.
        \end{law}

        这是一个原理, 这意味着它不是通过推到得出来的. 就如同电磁学中的麦克斯韦尔方程式, 统计热力学中的各态历经原理一样, 它们的正确性是由它们的推论的正确性保证的. 

        \subsection{Lagrange力学}

        接着我们来讨论有关与泛函和变分的问题. 在这里我并不想给出泛函与泛函微分在数学上严格的定义, 读者可以自行寻找相关参考书籍. 泛函可以认为是一个函数空间到$\mathbb{R}$的映射, 它的自变量是一个函数. 如果我们假设系统真实运动轨迹是$\bm q_\mathrm{c}(t)$, 显然有$\bm q_\mathrm{c}(t_1) = \bm q^{(1)}$, $\bm q_\mathrm{c}(t_2) =\bm q^{(2)}$. 则我们考虑一个对真实运动轨迹的一个微小偏离$\delta \bm q(t)$, 且在这里我们要求这个微小的偏离满足$\delta \bm q(t_1)=\delta \bm q(t_2)=0$. 这个对真实轨迹的微小偏离在数学上称为变分.

        最小作用量原理要求$\delta S = 0$, 那么我们首先需要根据定义写下$\delta S$:
        \begin{equation}
            \delta S = \int_{t_1}^{t_2} L( \bm q_\mathrm{c} - \delta \bm q(t_1) , \bm \dot q_\mathrm{c} - \delta \bm \dot q(t_1), t ) \dd t - \int_{t_1}^{t_2} L( \bm q_\mathrm{c} , \bm \dot q_\mathrm{c}, t ) \dd t
        \end{equation}
        使得$\delta \bm q$为无穷小量, 则可以得到\footnote{这里使用了爱因斯坦求和规定, 即隐含一个对相同指标$i$的求和号. 笔者认为这种表达比使用向量表示更不容易出错, 缺点在于指标太多的时候可能没法很好地把式子整理得很好看. (虽然以笔者的微レ存的数理基础, 用向量表示可能会推错式子)}:
        \begin{equation}
            \delta S = \int_{t_1}^{t_2} \dd t \left( \frac{\partial L}{\partial q_i} \delta q_i + \frac{\partial L}{\partial \dot q_i} \delta \dot q_i \right)
        \end{equation}
        我们注意到, 通过分步积分:
        \begin{equation}
            \frac{\partial L}{\partial \dot q_i} \delta \dot q_i = \frac{\mathrm{d}}{\mathrm{d} t} \left( \frac{\partial L}{\partial \dot q_i} \delta q_i \right) - \left[\frac{\mathrm{d} }{\mathrm{d} t} \frac{\partial L}{\partial \dot q_i} \right] \delta q_i
        \end{equation}
        可以得到:
        \begin{equation}\label{eq:8-1-1}
            \delta S = \left. \frac{\partial L}{\partial \dot q_i} \delta q_i \right|_{t_1}^{t_2} + \int_{t_1}^{t_2} \dd t \left( \frac{\partial L}{\partial q_i} - \frac{\mathrm{d} }{\mathrm{d} t} \frac{\partial L}{\partial \dot q_i} \right) \delta q_i = 0
        \end{equation}

        在端点处我们要求过$\delta \bm q(t_1)=\delta \bm q(t_2)=0$, 所以上式中第一项为0. 由于第二项对于任意的$\delta q_i(t)$都必须成立, 而且各个$\delta q_i(t)$都是完全独立的变分, 所以唯一的可能就是上式中的括号等于0. 于是我们推出了在Lagrange力学中描述运动的方程, Euler-Lagrange方程: 
        \begin{equation}
            \frac{\partial L}{\partial q_i} - \frac{\mathrm{d} }{\mathrm{d} t} \frac{\partial L}{\partial \dot q_i} = 0
            \qquad i = 1,2,\cdots,f.
        \end{equation}
        其中$f$为系统的自由度. 

        我们看到在Lagrange力学中拉格朗日量处于中心地位, 它描述了体系全部的经典力学性质. 对于非相对论的力学体系, 拉格朗日量可以写为动能减去势能$L = K - V$. 在一些坐标系下动能可能有些非常繁杂的表达式, 而势能仅依赖于广义坐标与时间. 我们可以把它写下来:
        \begin{equation}
            L =  \frac{M_{ij}(\bm q)}{2} \dot q_i \dot q_j  - V(\bm q, t)
        \end{equation}
        我们可以定义广义动量为: 
        \begin{equation}
            p_i =\frac{\partial L}{\partial \dot q_i} = M_{ij}(\bm q) \dot q_j
        \end{equation}

        \splitline

        考虑拉格朗日量对时间的全导数:
        \begin{equation}
            \frac{\mathrm{d} L}{\mathrm{d} t} = \frac{\partial L}{\partial t} + p_i \ddot q_i + \dot p_i \dot q_i = \frac{\partial L}{\partial t} +  \frac{\mathrm{d} (p_i \dot q_i)}{\mathrm{d} t}
        \end{equation}
        如果拉格朗日量不显含时间, 那么定义能量$E = p_i \dot q_i - L$是系统的守恒量. 
        受到系统能量的启发, 对拉格朗日量进行勒让德变换, 定义哈密顿量为$H = p_i \dot q_i - L$. 
        由这个定义可以得出:
        \begin{equation}
            \mathrm{d} H = \dot q_i \mathrm{d} p_i - \dot p_i \mathrm{d} q_i - \frac{\partial L}{\partial t} \mathrm{d} t
        \end{equation}
        与哈密顿量的全微分进行比较, 可以得出:
        \begin{equation}
            \dot{q_{i}} = \pdv{H}{p_i};
            \qquad
            \dot{p_{i}} = -\pdv{H}{q_i}
        \end{equation}
        即哈密顿正则方程.

        \subsection{Hamilton-Jacobi方程}

        我们有了作用量$S$这一个新的物理量, 我们可以研究一下其对广义坐标$\bm q$的偏导、对时间$t$的偏导与全导. 这里的偏导数非常地多, 注意弄清楚每个偏导中那些变量是不变的. 

        考虑对于广义坐标的$q$的偏导, 这里实际上考虑的是对于真实轨迹的初始位置$\bm q^{(1)}$与结束为止$\bm q^{(2)}$的偏导. 通过式\ref{eq:8-1-1}, 由于运动轨迹为真实轨迹, 所以第二项积分为0. 因此可以得到: 
        \begin{equation}
            \left( \frac{\partial S}{\partial q_i^{(1)}} \right)_{t, q_i^{(2)}} = - \frac{\partial L}{\partial \dot q_i};
            \qquad
            \left(\frac{\partial S}{\partial q_i^{(2)}}\right)_{t, q_i^{(1)}} = \frac{\partial L}{\partial \dot q_i}
        \end{equation}
        一般我们取的是末端点的偏导, 并且我们注意到拉格朗日量对广义速度的偏导数就是广义动量. 因此在不引起歧义时我们一般会写: 
        \begin{equation}
            \frac{\partial S}{\partial q_i} = p_i
        \end{equation}

        实际上作用量$S$形式上是一个变上限定积分, 因此全导非常容易得到: 
        \begin{equation}
            S[\bm q](t) = \int_{0}^{t} L( \bm q(\tau), \bm \dot q(\tau), \tau ) \dd \tau
            \qquad \implies \frac{\mathrm{d} S}{\mathrm{d} t} = L
        \end{equation}
        考虑到全导与偏导的关系, 有:
        \begin{equation}
            \frac{\mathrm{d} S}{\mathrm{d} t} = \frac{\partial S}{\partial t} + \frac{\partial S}{\partial q_i} \dot{q}_i = \frac{\partial S}{\partial t} + p_i\dot{q}_i
        \end{equation}
        那么根据哈密顿量的定义$H = p_i \dot q_i - L$, 可以看出:
        \begin{equation}\label{eq:8-1-2-1}
            \frac{\partial S}{\partial t} = - H
        \end{equation}

        基于上面的结论, 我们发现只使用作用量$S$及其偏导也可以表述经典力学: 
        \begin{equation}
            \frac{\partial S}{\partial t} + H(\bm q, \frac{\partial S}{\partial \bm q}, t) = 0
        \end{equation}
        进一步展开可以写为:
        \begin{equation}\label{eq:8-1-2-2}
            \frac{\partial S}{\partial t} + \frac{M^{-1}_{ij}(\bm q)}{2} \frac{\partial S}{\partial q_i}\frac{\partial S}{\partial q_j} + V(\bm q, t) = 0
        \end{equation}
        $M^{-1}_{ij}$是质量矩阵的逆. 

        以上\ref{eq:8-1-2-1}至\ref{eq:8-1-2-2}三个方程都可以被称为Hamilton-Jacobi方程. 它提供了一种只使用作用量$S$的偏导数表达经典力学体系的运动规律的方法.

        \subsection{我们可以更深入一些}

        此处应有讨论边界值问题解的存在性相关的内容.

    \section{量子力学的路径积分形式}

        \subsection{传播子与路径积分}

        我们基于含时薛定谔方程已经得到了时间演化算符, 当不同时刻的哈密顿算符对易且不显含时间时:
        \begin{equation}
            | \psi(t) \rangle = \mathrm{e}^{-\frac {\mathrm{i}\hat{H}t}{\hbar}} | \psi(0) \rangle
        \end{equation}
        插入位置本征态:
        \begin{equation}
            \langle x | \psi(t) \rangle = \int \mathrm{d} y \langle x | \mathrm{e}^{-\frac {\mathrm{i}\hat{H}t}{\hbar}} | y \rangle  \langle y | \psi(0) \rangle
        \end{equation}
        我么可以将$\langle x | \exp[-\mathrm{i}\hat{H}t / \hbar] | y \rangle$定义为传播子, 如果求出传播子, 那么就可以求解含时Schrodinger方程. 这时研究对象就从波函数变为传播子, 现在需要求解传播子. 

        首先研究$\mathrm{e}^{\lambda \hat{A}}\mathrm{e}^{\lambda \hat{B}}$和$\mathrm{e}^{\lambda (\hat{A}+\hat{B})}$的关系. 根据Baker–Campbell–Hausdorff公式, 应有:
        \begin{equation}
            \mathrm{e}^{\lambda \hat{A}} \mathrm{e}^{\lambda \hat{B}} = \mathrm{e}^{\lambda (\hat{A}+\hat{B}) + \frac 12 \lambda^2 [\hat{A},\hat{B}] + O(\lambda^2)}
        \end{equation}
        如果$\lambda \to 0$, 可以忽略二阶无穷小量, 则有:
        \begin{equation}
            \mathrm{e}^{\lambda \hat{A}} \mathrm{e}^{\lambda \hat{B}} = \mathrm{e}^{\lambda (\hat{A}+\hat{B})}
        \end{equation}
        将时间平均分为$N$份, 令$\lambda = \frac tN$, 并令$N \to \infty$. 所以:
        \begin{equation}
            \mathrm{e}^{-\frac {\mathrm{i}\hat{H}t}{N\hbar}} = \mathrm{e}^{-\frac {\mathrm{i}\hat{K}t}{N\hbar}} \mathrm{e}^{-\frac {\mathrm{i}\hat{V}t}{N\hbar}}
        \end{equation}
        代入传播子, 得到:
        \begin{equation}\begin{aligned}
            \langle x | \mathrm{e}^{-\frac {\mathrm{i}\hat{H}t}{N\hbar}} | y \rangle &= \langle x | \mathrm{e}^{-\frac {\mathrm{i}t}{N\hbar}\frac {\hat{p}^2}{2m}} 
            \mathrm{e}^{-\frac {\mathrm{i}t}{N\hbar}\hat{V}} |y \rangle\\
            &= \langle x | \mathrm{e}^{- \frac{\mathrm{i}t}{N\hbar} \frac{\hat{p}^2}{2m}} |y \rangle \mathrm{e}^{-\frac {\mathrm{i}t}{N\hbar}V(y)} \\
            &= \mathrm{e}^{-\frac {\mathrm{i}t}{N\hbar}V(y)} \int \langle x | \mathrm{e}^{-\frac {\mathrm{i}t}{N\hbar}\frac {\hat{p}^2}{2m}} |p \rangle \langle p |y \rangle \mathrm{d}p \\
            &= \mathrm{e}^{-\frac {\mathrm{i}t}{N\hbar}V(y)} \int \langle x|p \rangle \langle p|y \rangle \mathrm{e}^{-\frac {\mathrm{i}t}{N\hbar}\frac {p^2}{2m}} \mathrm{d}p\\
            &= \frac 1{2\pi \hbar} \mathrm{e}^{-\frac {\mathrm{i}t}{N\hbar}V(y)} \int \mathrm{e}^{\frac {\mathrm{i}(x-y)p}{\hbar}} \mathrm{e}^{-\frac {\mathrm{i}t}{N\hbar}\frac {p^2}{2m}} \mathrm{d} p
        \end{aligned}\end{equation}
        根据Gauss积分
        \begin{equation}
            \int_{-\infty}^{+\infty} \mathrm{e}^{-ax^2+bx} \mathrm{d}x = \sqrt{\frac {\pi}a} \mathrm{e}^{-\frac {b^2}{4a}}
        \end{equation}
        由此得到传播子为:
        \begin{equation}
            \langle x | \mathrm{e}^{-\frac {\mathrm{i}\hat{H}t}{N\hbar}} | y \rangle = \sqrt{\frac {mN}{2\pi\mathrm{i} \hbar t}} \mathrm{e}^{\mathrm{i}\frac {mN(x-y)^2}{2\hbar t}}\mathrm{e}^{-\frac {\mathrm{i}t}{N\hbar} V(y)}
        \end{equation}
        前两项来自于动能算符, 第三项来自于势能算符. 动能算符和势能算符虽然不对易, 但是在$N \to \infty$时可以得到这个结果. 

        \splitline

        而一般情况下的传播子可以通过对无穷短时间的传播子积分得到:
        \begin{equation}
            \langle x_0 | \mathrm{e}^{-\frac {\mathrm{i}\hat{H}t}{\hbar}} | x_N \rangle = \left(\frac{mN}{2\pi\mathrm{i}\hbar t}\right)^{N/2} \left(\prod_{n = 1}^{N}\int 
            \mathrm{d} x_n \right) \mathrm{e}^{ \frac{\mathrm{i}t}{\hbar N} \sum_{n = 1}^{n} \left[ \frac{m}{2} \left(\frac{x_{n-1} - x_n}{t/N}\right)^2 - V(x_n)\right] }
        \end{equation}
        在$N \to \infty$的情形下$N(x_{n-1} - x_n)/t = \dot x_n$, 且可将求和号换为积分号. 那么指数上可写为: 
        \begin{equation}\begin{aligned}
            \frac{\mathrm{i}t}{\hbar N} \sum_{n = 1}^{n} \left[ \frac{m}{2} \left(\frac{x_{n-1} - x_n}{t/N}\right)^2 - V(x_n)\right] &= \frac{\mathrm{i}}{\hbar} \int_0^t \mathrm{d}t \left[ \frac{1}{2} m \dot x^2 - V(x) \right]\\
            &= \frac{\mathrm{i}}{\hbar} \int_0^t \mathrm{d}t L = \frac{\mathrm{i}}{\hbar} S_t[x]
        \end{aligned}\end{equation}
        作为一种表示方法, 有: 
        \begin{equation}\label{eq:8-2-1}
            \langle x_0 | \mathrm{e}^{-\frac {\mathrm{i}\hat{H}t}{\hbar}} | x_N \rangle = \int \mathcal{D}[x] \mathrm{e}^{ \frac{\mathrm{i}}{\hbar} S_t[x] }
        \end{equation}
        其中$\mathcal{D}[q]$表示对所有可能的路径进行积分, 同时将归一化因子作为测度包含在内(这个归一化因子应当是与路径无关的). 我们发现传播子其实是一种对所有路径进行加权平均. 其中所有路径权重的模相等, 但具有与作用量相关的相位. 由于作用量的形式与经典作用量一致, 所以只要我们能算得出这个积分就可以求得传播子. 

        注意虽然上面似乎给出了归一化因子的一个形式, 但很明显它对于$N \to \infty$不是一个好的定义. 实际上不同体系归一化因子并不是一样的, 需要通过实际计算确定. 

        \subsection{自由粒子的传播子}

        自由粒子体系的势能为0, 所以可以不需要把时间分成$N$份, 而是直接对整个传播子来计算. 把$V=0,N=1$代入上式, 即得到:
        \begin{equation}\label{eq:8-2-2}
            \langle x_0 | \mathrm{e}^{-\frac {\mathrm{i}\hat{H}t}{\hbar}} | y_0 \rangle = \sqrt{\frac {m}{2\pi\mathrm{i} \hbar t}} \mathrm{e}^{-\mathrm{i}\frac {m(x_0 - y_0)^2}{2\hbar t}}
        \end{equation}
        如果推广到$F$维体系, 则有:
        \begin{equation}
            \langle \bm{x_0} | \mathrm{e}^{-\frac {\mathrm{i}\hat{H}t}{\hbar}} |\bm{y_0} \rangle = (\frac {1}{2\pi\mathrm{i} \hbar t})^{\frac F2} |\mb{M}|^{\frac 12} \mathrm{e}^{-\mathrm{i}\frac {(\bm{x_0}-\bm{y_0})^{\mathrm{T}} \mb{M} (\bm{x_0}-\bm{y_0})}{2\hbar t}}
        \end{equation}

        \splitline

        考虑直接通过路径积分得到自由粒子的传播子. 在经典情况下写出自由粒子作用量:
        \begin{equation}
            S_t[x] = \int_0^t \mathcal{L}(x,\dot{x},t') \mathrm{d}t' = \frac 12 \int_0^t m\dot{x}^2 \mathrm{d}t'
        \end{equation}
        Lagrange函数会满足Euler-Lagrange方程. 而对于自由粒子, Lagrange函数不显含坐标, 所以:
        \begin{equation}
            \frac {\mathrm{d}}{\mathrm{d}t} (m \dot x) = 0
        \end{equation}
        由此可见, 速度不随时间变化, 且:
        \begin{equation}
            \dot{x} = \frac{y_0 - x_0}t
        \end{equation}
        所以, 上述作用量积分的结果为:
        \begin{equation}
            S = \frac{m(y_0 - x_0)^2}{2t}
        \end{equation}
        由于自由粒子的作用量与路径毫无关系, 利用传播子的一般形式(式\ref{eq:8-2-1})很容易算出传播子的表达式:
        \begin{equation}
            \langle x_0 | \mathrm{e}^{-\frac {\mathrm{i}\hat{H}t}{\hbar}} | y_0 \rangle = \int \mathcal{D}[x] \mathrm{e}^{ \frac{\mathrm{i}}{\hbar} S_t[x] } = C(t) \mathrm{e}^{\frac {\mathrm{i}}{\hbar} \frac{m(y_0 - x_0)^2}{2t}}
        \end{equation}
        其中$C(t)$是归一化因子. 现在希望能把$C(t)$求出, 给定初始条件:
        \begin{equation}
            t \to 0,~~~~~\langle x_0|y_0\rangle = \delta(y_0-x_0)
        \end{equation}
        计算出:
        \begin{equation}
            \int_{-\infty}^{+\infty} \mathrm{e}^{\frac {\mathrm{i}}{\hbar} \frac {m(y_0 - x_0)^2}{2t}}\mathrm{d}y_0 = \sqrt{\frac {2\pi\mathrm{i}\hbar t}m}
        \end{equation}
        于是:
        \begin{equation}
            C(t) = \sqrt{\frac m{2\pi\mathrm{i}\hbar t}}
        \end{equation}
        我们已经得出了归一化因子. 

        如果不放心, 可设传播子为$ C(t) D(t) \mathrm{e}^{ \frac{\mathrm{i}}{\hbar} S } $, 带回含时薛定谔方程验证($D(0) = 1$).
        \begin{equation}
            -\mathrm{i}\hbar \frac {\partial}{\partial t} \langle y_0 | \mathrm{e}^{-\frac {\mathrm{i}\hat{H}t}{\hbar}} |x_0 \rangle =  \langle y_0 | \hat{H} \mathrm{e}^{-\frac {\mathrm{i}\hat{H}t}{\hbar}} |x_0 \rangle = -\frac {\hbar^2}{2m} \frac {\partial^2}{\partial y_0^2} \langle y_0 | \mathrm{e}^{-\frac {\mathrm{i}\hat{H}t}{\hbar}} |x_0 \rangle
        \end{equation}
        展开之后可以得到$D'(t) = 0$, 所以$D(t)=1$是一个常数函数. 
        \begin{equation}\begin{aligned}
            \left[i\hbar D'(t) C(t) - \frac{i\hbar}{2t} D(t) C(t) + \frac{m(x_0-y_0)^2}{2t^2} D(t) C(t) \right] \mathrm{e}^{\frac {\mathrm{i}}{\hbar} S} \\
            = \left[- \frac{i\hbar}{2t} D(t) C(t) + \frac{m(x_0-y_0)^2}{2t^2} D(t) C(t)\right] \mathrm{e}^{\frac {\mathrm{i}}{\hbar} S}
        \end{aligned}\end{equation}
        由此可以确定自由粒子的传播子, 结果与式\ref{eq:8-2-2}相吻合. 

        \subsection{其他体系的传播子}

        即使不是自由体系也可以利用式\ref{eq:8-2-1}进行计算, 但能计算出解析形式传播子的体系非常少. 目前, 传播子可以写出解析表达式的体系只有: 自由粒子、无限深方势阱、谐振子、Morse势、类氢原子\cite{Grosche_HFPI}\cite{Kleinnert_PIQMSPP}. 

        谐振子体系是一个可以求出解析形式传播子的体系, 读者可以尝试计算一下谐振子的传播子作为练习. 结果为:
        \begin{equation}
            \langle x | \mathrm{e}^{-\frac {\mathrm{i}}{\hbar}\hat{H}_\mathrm{HO}t} | y \rangle = \sqrt{\frac{m\omega}{2\mathrm{i}\pi\hbar\sin(\omega t)}} \exp\left\{ \frac{\mathrm{i}}{\hbar} \frac{m\omega}{2\sin(\omega t)}\left[ (x^2 + y^2)\cos(\omega t) - 2xy \right] \right\} \\
        \end{equation}

        有关这方面的内容, 可以参考\cite{费曼量子力学与路径积分}\cite{谷村吉隆化学物理入门}.

        \begin{asg}
            第8次作业第四题
        \end{asg}

    \section{路径积分分子动力学}

        我们注意到量子体系下的Boltzmann分布为
        \begin{equation}
            \mathrm{e}^{-\beta \hat{H}} = \sum_n \mathrm{e}^{-\beta E_n} |\phi_n \rangle \langle \phi_n|
        \end{equation}
        利用了Schodinger方程
        \begin{equation}
            \hat{H} |\phi_n \rangle = E_n |\phi_n \rangle
        \end{equation}
        类似地, 可以作和
        \begin{equation}
            \mathrm{e}^{-\frac {\mathrm{i}\hat{H}t}{\hbar}} = \sum_n \mathrm{e}^{-\frac {\mathrm{i}E_n t}{\hbar}} |\phi_n \rangle \langle \phi_n|
        \end{equation}
        上述两个式子可以对应起来, 区别在于第二个方程中的时间是虚数, 称为\textbf{虚时间}. 对应关系为
        \begin{equation}
            t = -\mathrm{i}\hbar \beta
        \end{equation}
        显然地, 高温对应虚时间的短时, 低温对应虚时间的长时. 同样可求出虚时间下的传播子
        \begin{equation}
            \langle x|\mathrm{e}^{-\beta \hat{H}}|y\rangle
        \end{equation}
        求出配分函数
        \begin{equation}\begin{aligned}
            Z &= \mathrm{Tr}(\mathrm{e}^{-\beta \hat{H}}) \\
            &= \sum_n \langle n| \mathrm{e}^{-\beta \hat{H}}|n\rangle \\
            &= \int \sum_n \langle n| \mathrm{e}^{-\beta \hat{H}}|x \rangle \langle x|n\rangle \mathrm{d}x\\
            &= \int \langle x|\sum_n|n \rangle \langle n| \mathrm{e}^{-\beta \hat{H}}|x\rangle \mathrm{d}x\\
            &= \int \langle x|\mathrm{e}^{-\beta \hat{H}}|x\rangle \mathrm{d}x
        \end{aligned}\end{equation}
        在路径积分的语言下, 可以放弃态的概念, 也不需要有波函数, 只要有传播子, 就有配分函数, 也就有了所有的热力学函数. 要求这个积分中的传播子, 将$\beta$分为$N$份.有
        \begin{equation}
            \langle x|\mathrm{e}^{-\beta \hat{H}}|x\rangle  = \int \langle x_0|\mathrm{e}^{-\frac {\beta \hat{H}}N} |x_1 \rangle ... \langle x_{n-2}|\mathrm{e}^{-\frac {\beta \hat{H}}N} |x_{n-1} \rangle \langle x_{n-1}|\mathrm{e}^{-\frac {\beta \hat{H}}N} |x_N \rangle \prod_{i=1}^{N-1}\mathrm{d}x_i
        \end{equation}
        其中$x_0=x_N = x$. 如果$N \to \infty$, 则可以把动能项和势能项分开. 用和之前传播子计算类似的方法得到
        \begin{equation}
            \langle x_j|\mathrm{e}^{-\Delta \beta \hat{H}} |x_{j+1} \rangle = \sqrt{\frac m{2\pi \hbar^2 \Delta \beta}} \mathrm{e}^{-\frac {m(x_j- x_{j+1})^2}{2\hbar^2 \Delta \beta}} \mathrm{e}^{-\Delta \beta V(x_{j+1})}
        \end{equation}
        定义$\omega_N^2 = \frac N{\hbar^2 \Delta \beta^2} = \frac N{\hbar^2 \beta^2}$, 则有
        \begin{equation}
            \langle x_j|\mathrm{e}^{-\Delta \beta \hat{H}} |x_{j+1} \rangle = \sqrt{\frac m{2\pi \hbar^2 \Delta \beta}} \mathrm{e}^{-\frac {\beta m\omega_N^2(x_j- x_{j+1})^2}{2}} \mathrm{e}^{-\Delta \beta V(x_{j+1})}
        \end{equation}
        代入得到 
        \begin{equation}
            \langle x|\mathrm{e}^{-\beta \hat{H}}|x\rangle  = \bigg(\frac {mN}{2\pi \hbar^2 \beta}\bigg)^{\frac N2} \int \mathrm{e}^{-\sum_{j=0}^{N-1} \frac {\beta}2 m \omega_N^2 (x_{j+1} - x_j)^2} \mathrm{e}^{-\Delta \beta \sum_{j=0}^{N-1}V(x_j)}\prod_{i=1}^{N-1}\mathrm{d}x_i
        \end{equation}
        这可以看作$N$个点组成的环两两用弹簧连接, 且每个点都额外受外力作用. 它可以写成
        \begin{equation}
            \langle x|\mathrm{e}^{-\beta \hat{H}}|x\rangle  = \int C(N) \mathrm{e}^{-\beta V_{\mathrm{eff}}(\bm{x})} \mathrm{d}\bm{x}
        \end{equation}
        积分不太容易做, 可以在这里插入一个关于“动量”的积分
        \begin{equation}\begin{aligned}
            \langle x|\mathrm{e}^{-\beta \hat{H}}|x\rangle  &= \int D(N) \mathrm{e}^{-\beta V_{\mathrm{eff}}(\bm{x})} \mathrm{d}\bm{x} \int \mathrm{d}\bm{p} \mathrm{e}^{-\frac {\beta}2 \bm{p}^{\mathrm{T}}\bm{M}^{-1} \bm{p}}\\
            &= \int D(N) \mathrm{e}^{-\beta H_{\mathrm{eff}}(\bm{x,p})} \mathrm{d}\bm{x} \mathrm{d}\bm{p}
        \end{aligned}\end{equation}
        如果$N=1$, 那么就是经典统计力学的结果. 这体现了量子力学和经典统计力学的同构. 

        如果关心含时Schodinger方程:
        \begin{equation}
            \mathrm{i}\hbar \frac {\partial}{\partial t} | \psi(t) \rangle = \hat{H}|\psi(t) \rangle
        \end{equation}
        将这个改成对于$\beta$的方程:
        \begin{equation}
            -\frac {\partial}{\partial \beta} |\psi(\beta) \rangle = \hat{H} |\psi(\beta) \rangle
        \end{equation}

        任意给定一个初态, 可以写成Hamilton函数的本征函数的线性组合
        \begin{equation}
            |\psi(0) \rangle = \sum_n c_n |\phi_n\rangle
        \end{equation}
        并且
        \begin{equation}
            |\psi(\beta) \rangle = \mathrm{e}^{-\beta \hat{H}} |\phi(0) \rangle = \sum_n c_n\mathrm{e}^{-\beta E_n}|\phi_n \rangle = \mathrm{e}^{-\beta E_0} \sum_n c_n \mathrm{e}^{-\beta(E_n-E_0)} |\phi_n \rangle
        \end{equation}
        因此, 当$\beta \to \infty$时, 得到的就是基态. 基于此开发出了\textbf{量子Monte-Carlo算法}
        如果要求第一激发态, 只需求出系数, 把基态从初始条件中减去, 得到的新的最低的能量对应的态就是第一激发态. 

    \bibliographystyle{plain}
    \bibliography{ref_chp_8}