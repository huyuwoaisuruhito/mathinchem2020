\chapter{Liouville 方程}
    \section{Liouville方程}
    \subsection{有关系综的概念}\footnote{这一节的更多内容可以参考\cite{Tuckerman2010Statistical}
    和刘川老师的平衡态统计物理讲义第二章开头}
    统计物理的目标是建立宏观物理量和微观运动规律之间的联系,之前讨论的Hamilton方程可以用来描述经典意义下
    微观系统的运动。某个微观系统的运动状态(某一时刻各个粒子的广义坐标和广义动量)对应相空间中一个点(
    称为系统的\textbf{代表点}),系统随时间的演化等价于代表点在相空间中按照Hamilton方程决定的轨线运动。Maxwell
    与Boltzmann认为,对于宏观系统的测量发生在一段时间$(t_0, t_0 + \tau)$内,其中$\tau$是一个
    宏观短(指系统的宏观物理量没有发生可观测的变化)、微观长(指在这段时间内系统的代表点在相空间中发生了
    很明显的移动)的时间段,而真正观测到的物理量$A(t_0)$是对应微观物理量$a(x,p)$的\textbf{时间平均}:
    \begin{equation}
        A(t_0) := \frac{1}{\tau}\int_{t_0}^{t_0+\tau}a(x(t), p(t))\dd x\dd p
    \end{equation}
    但是我们无法求解由大量粒子(~$10^{23}$)构成的复杂系统的运动方程,无法给出系统的相轨道,
    因此上面的定义难以给出有意义的结论。
    \par 
    Boltzmann通过提出\textbf{各态历经假设}
    \footnote{从数学上讲各态历经假设并不是对于任意力学系统都成立}
    来解决上面遇到的困难,Boltzmann认为对于能量守恒的系统,
    经过足够长(微观上)时间演化后,系统的代表点在等能面上每一个点邻域内都停留相同长的时间。
    利用这种思想,我们可以定义系统在宏观短微观长的时间内在相空间中每一点附近出现的概率$\rho(x, p, t)$,
    那么就能将物理量对时间的平均转化到对相空间的平均:
    \begin{equation}
        A(t_0) = \int\rho(x, p, t_0)a(x, p)\dd x\dd p
    \end{equation}
    随着我们引入相空间中的概率密度函数$\rho(x, p, t)$,我们已经更换了思考问题的角度。我们不再考虑\textbf{一个
    系统}长时间演化过程中在相空间中的分布,而是直接考虑以同样的密度函数分布在相空间中的\textbf{大量完全相同的系统},
    待求的宏观物理量是对应的微观物理量在这些系统上的平均值,可以想象,如果各态历经假设是成立的,那么这两种
    平均应该给出相同的结果。
    \par 
    有了上面的讨论,就将求出任意时刻热力学量的问题转化为了求出任意时刻密度函数$\rho(x, p, t)$的问题,
    这个密度函数被称为\textbf{系综密度函数},下面给出系综的准确定义。
    满足同样宏观条件(如能量、体积、粒子数)的系统有可能处于不同的微观运动状态、对应相空间中不同的代表点
    ,将具有\textbf{相同宏观条件}的所有系统
    \footnote{这些系统必须是“相同的系统”,即是用Hamilton方程所描述的系统,形象地说有着相同数量、种类
    的粒子,并且相互作用也相同;但是这里的相同并不是说运动状态也相同,它们对应着相空间中不同的代表点
    }
    的集合称为此宏观条件下的一个\textbf{系综}。可以定义系综内系统代表点在相空间的归一化密度函数$\rho(x, p, t)$,
    这个函数代表了这个系综内系统的代表点在$(x, p)$邻域出现的概率,
    满足:
    \begin{equation}
        \begin{split}
            &\rho(x, p, t) \geq 0\\
            &\int\rho(x, p, t)\dd x\dd p = 1
        \end{split}
        \label{ensemble density}
    \end{equation}
    宏观系统(宏观物理量)的时间演化用系综随时间的演化来描述:系综中的每一个系统代表点都按照Hamilton方程
    在相空间中运动,代表点会随着时间在相空间中重新分布,对应的系综密度函数也会随时间变化
    \footnote{既然系综内系统的代表点按照Hamilton方程运动,这样引起的系综密度的变化是完全确定的,
    可以用下一节将要介绍的Liouville方程描述}
    ,任意时刻的物理量按照对应时刻的系综密度函数计算。
    \subsection{Liouville方程}
    \footnote{这一节的更多内容可以参考\cite{Tuckerman2010Statistical2}}
    考虑给定初始时刻$t=0$大量系统的代表点在相空间中的一个分布
    \footnote{这样的分布可能是某个系综对应的分布,也有可能只是任意给定的分布,不代表真实的系综}
    ,这个分布可以用一个归一化的密度函数$\rho(x, p, 0)$(满足\ref{ensemble density})来描述,
    这些系统的代表点将在相空间中按照Hamilton方程演化,我们想知道$t$时间后相空间中代表点的密度函数,
    即希望给出$\rho(x, p, 0)$满足的方程。
    \par
    假设给定的分布在相空间中代表点的总数目为$N$,对于0时刻任意相空间中的区域$D$
    \footnote{在我们的讨论中总假设$D$是有体积的},其中代表点的数目$n(D)$为:
    \begin{equation}
        n(D) = N\int_{D}\rho(x, p, 0)\dd x\dd p
    \end{equation}
    考虑$D$内所有点都按照Hamilton方程在相空间中运动,那么$t$时刻$D$将演化为$\phi^t(D)$
    \footnote{$\phi^t$是Hamilton相流中的一个元素,具体定义参见前一章关于相流的讨论},
    在演化过程中,$D$内的代表点不会从中“跑出”,也不会有新的代表点进入,更不会凭空消失,
    因此$\phi^t$内的代表点数目维持不变,即:
    \begin{equation}
        n(\phi^t(D)) = N\int_{\phi^t}\rho(x, p, t)\dd x\dd p = n(D)
    \end{equation}
    那么可以得到:
    \begin{equation}
        \int_{D}\rho(x, p, 0)\dd x\dd p = \int_{\phi^t(D)}\rho(x, p, t)\dd x\dd p
    \end{equation}
    对等式右边应用重积分的换元,利用$\phi^t$将$\phi^t(D)$变换为0时刻的区域$D$(参考\ref{volum of D}),
    同时利用Liouville定理,上面的等式可转化为:
    \begin{equation}
        \int_{D}\rho(x_0, p_0, 0)\dd x_0\dd p_0 = \int_{D}\rho(x_t(x_0, p_0), p_t(x_0, p_0), t)\dd x_0 \dd p_0
    \end{equation}
    上式中$(x_t, p_t)$的精确含义是$(x_t, p_t) = \phi^t(x_0, p_0)$,表示$(x_t, p_t)$是由
    $(x_0, p_0)$按照Hamilton方程演化$t$时间后到达的点。由于上面的区域$D$是任意给定的,那么可以得到:
    \begin{equation}
        \rho(x_t(x_0, p_0), p_t(x_0, p_0), t) = \rho(x_0, p_0, 0)
        \label{Liouville equation Lagrange}
    \end{equation}
    对上面的等式对时间求导\footnote{这里的求导是沿着轨线进行的}:
    \begin{equation}
        \frac{\mathrm{d}\rho}{\mathrm{d}t} = \frac {\partial \rho}{\partial t} + \frac {\partial \rho}{\partial \bm{x}_t}\dot{\bm{x}}_t  + \frac {\partial \rho}{\partial \bm{p}_t} \dot{\bm{p}}_t = 0
    \end{equation}
    再利用正则方程,得到
    \begin{equation}
        - \frac {\partial \rho}{\partial t} = \bigg(\frac {\partial \rho}{\partial \bm{x}}\bigg)^\mathrm{t} \frac {\partial H}{\partial \bm{p}} - 
        \bigg(\frac {\partial \rho}{\partial \bm{p}}\bigg)^\mathrm{t} \frac {\partial H}{\partial \bm{x}}
        \label{Liouville equation Euler}
    \end{equation}
    这个方程被称为Liouville方程。
    \par 
    考虑定义在相空间中的两个函数$A(\bm{x}, \bm{p}),B(x ,p)$,定义这两个函数的\textbf{Poisson括号}
    \footnote{
        可以使用更紧凑的形式表达Poisson括号。考虑Hamilton方程\ref{canonical equation},其中将$(\bm{x}, \bm{p})$
        合并为正则变量$\bm{\eta}$,基于这种考虑,可以将Poisson括号写为:
        \begin{equation}
            \{A, B\} = \left(\pdv{A}{\bm{\eta}}\right)^\mr{t}\bm{J}\left(\pdv{B}{\bm{\eta}}\right)
        \end{equation}
        从这种形式的Poisson括号更容易看出其是正则变换下的不变量。
    }
    :
    \begin{equation}
        \{A, B\} := \left(\pdv{A}{\bm{x}}\right)^\mr{t}\left(\pdv{B}{\bm{p}}\right) - \left(\pdv{A}{\bm{p}}\right)\mr{t}\left(\pdv{B}{\bm{x}}\right)
    \end{equation}
    利用Poisson括号重写Liouville方程:
    \begin{equation}
        - \frac {\partial \rho}{\partial t} = \{ \rho, H\}
    \end{equation}
    得到Liouville方程的另一种表述形式。如果Hamilton函数满足形式:
    \begin{equation}
        H(\bm{x},\bm{p}) = \frac{1}{2}\bm{p}^\mathrm{t} \bm{M}^{-1} \bm{p} + V(\bm{x})
    \end{equation}
    那么这个系统的Liuville方程就可以写为:
    \begin{equation} 
        - \frac {\partial \rho}{\partial t} = \bigg(\frac {\partial \rho}{\partial \bm{x}}\bigg)^\mathrm{t} \bm{M}^{-1} \bm{p}
         - \bigg(\frac {\partial \rho}{\partial \bm{p}}\bigg)^\mathrm{t} \frac {\partial V}{\partial \bm{x}}
    \end{equation}
    \par
    这里给出一种常见的分布——\textbf{Boltzmann分布}\footnote{这是正则系综(NVT系综)的系综密度函数}:
    \begin{equation}
        \rho(\bm{x},\bm{p}) \propto \mathrm{e}^{-\beta H(\bm{x},\bm{p})}
    \end{equation}

    如果一个分布满足:
    \begin{equation}
        \frac {\partial \rho}{\partial t} = 0
    \end{equation}
    那么可以看出在相空间中任何一点$(\bm{x}_1, \bm{p}_1)$的系综密度$\rho(\bm{x}_1, \bm{p}_1, t)$
    是一个与时间无关的常量(此时系综密度函数不显含时间),在这种情况下,对于任意微观量$a(\bm{x}, \bm{p})$
    其系综平均值为:
    \begin{equation}
        A(t) = \int\rho(\bm{x}, \bm{p})a(\bm{x}, \bm{p})\dd \bm{x}\dd \bm{p}
    \end{equation}
    $A$是一个不随时间变化的微观量,那么称这样的分布为\textbf{稳态分布}。

    \section{求解Liouville方程}
    一般我们遇到的是给定初始系综密度分布,求解$t$时刻系综密度的问题,即如下一阶偏微分方程的初值问题
    (为了方便书写假设为一维系统):
    \begin{equation}
        \left\{
        \begin{split}
            -\pdv{\rho}{t} &= \pdv{\rho}{x}\pdv{H}{p} - \pdv{\rho}{p}\pdv{H}{x}\\
            \rho(x, p, 0) &= f(x, p)
        \end{split}
        \right.
    \end{equation}
    在求解这个问题时很难使用解析的方法,一般都是利用数值解法求解。针对Liouville方程的特点,我们
    从两个视角出发,给出两种基本的解法。
    从方程\ref{Liouville equation Euler}出发,可以研究给定区域(给定点附近)内系综密度随时间的变化,
    这称为\textbf{Euler图象}
    \footnote{这个名称来源于流体力学,Euler通过描述空间内的\textbf{流速场}来描述流体的运动;Lagrange通过描述
    每一个质点的\textbf{运动轨迹}来描述流体的运动。}
    。从方程\ref{Liouville equation Lagrange}出发,可以追踪相空间内按照Hamilton方程运动的点处的系综密度
    ,轨线上任意一点的系综密度值都可以通过初值确定,这称为\textbf{Lagrange图象}。
    \subsection{从Euler图像求解任意时刻的系综密度}
    为了更加生动地讲解方法,这里给出实际地例子。
    回顾第一章中使用经典力学描述HCl分子的振动,在那里只讨论了谐振子地情形,事实上可以用Morse势来
    更精确地描述这个振动,Morse势被定义为:
    \begin{equation}
        V(x) = D_e (1- \mathrm{e}^{-a(r-r_\mathrm{eq})})^2 = D_e(1-\mathrm{e}^{-ax})^2
    \end{equation}
    其中$a>0$, 在平衡位置附近可以使用谐振子近似(Taylor展开到二阶)。写出谐振子近似下的Boltzmann分布:
    \begin{equation}
        \rho(x,p,0) = \frac {1}{Z} \mathrm{e}^{-\beta (\frac {p^2}{2m} + \frac 12 m\omega^2 x^2)} 
    \end{equation}
    其中谐振子的参数$\omega$后面给出。
    上面给出的系综密度函数不满足归一化条件,只有归一化之后才会使严格意义上的系综密度函数,
    显然归一化常数$Z$为:
    \begin{equation}
        Z = \int\ee^{-\beta H(x, p)}\dd x\dd p
    \end{equation}
    这个归一化常数被称为\textbf{正则配分函数}。归一化的过程涉及到Gauss函数的积分,下面给出一个一般的定义:
    \begin{equation}
        I(a) = \int_0^{+\infty} \mathrm{e}^{-ax^2} x^{n} \mathrm{d}x
        \label{1_D Gauss integral}
    \end{equation}
    \footnote{
        对于最基本的Gauss积分:
        \begin{equation}
            \int_{0}^{+\infty}
        \end{equation}
    }

    令$t = ax^2$, 则$\mathrm{d}t = 2ax\mathrm{d}x$
    所以
    \begin{equation*}
    I &= \int_0^{+\infty} \mathrm{e}^{-t} \bigg(\frac ta\bigg)^{\frac n2} \frac {\mathrm{d}t}{\sqrt{at}}\\
    &= \frac 1{2a^{\frac {n+1}2}} \int_0^{+\infty} \mathrm{e}^{-t} t^{\frac {n-1}2} \mathrm{d}t\\
    &= \frac {\Gamma(\frac {n+1}2)}{2a^{\frac {n+1}2}}
    \end{equation*}
    据此算出配分函数
    \begin{equation*}
        Z = \int \mathrm{e}^{-\beta (\frac {p^2}{2m} + \frac 12 m\omega^2 x^2)} \mathrm{d}x\mathrm{d}p = \frac {2\pi}{\beta \omega}
    \end{equation*}
    从量纲上分析,在配分函数中少了$\mathrm{d}x\mathrm{d}p$的量纲。本质上应该除以$2\pi\hbar$, 相当于对相空间做了量子化。于是
    \begin{equation*}
        Z = \frac 1{\beta \hbar \omega}
    \end{equation*}
    就是无量纲的配分函数。

    回到用Morse势描述HCl的振动的问题,Morse势的常数$a$可以用谐振子近似的$\omega$进行估计。令$x \to 0 $,对$V(x)$在平衡位置附近作Taylor展开,展开到二阶。
    \begin{equation*}
        V(x) = D_e a^2 x^2 + o(x^2)
    \end{equation*}
    它与谐振子近似一致,因此
    \begin{equation*}
        \frac 12 m\omega^2x^2 = D_e a^2 x^2
    \end{equation*}
    于是
    \begin{equation*}
        \omega = \sqrt{\frac {2D_ea^2}m}
    \end{equation*}
    事实上,对双原子分子HCl, 它有6个自由度,3个平动,2个转动,所以我们可以只用振动自由度来描述HCl的分子结构。

    \subsection{20201016:Lagrange图象演化概率密度}
    除了用Euler图象来演化密度以外,也可以用Lagrange图象来演化密度。由
    \begin{equation*}
        \frac {\mathrm{d}}{\mathrm{d}t} \rho(x_t,p_t,t) = 0
    \end{equation*}
    可以得到$t$时刻的概率密度为
    \begin{equation*}
        \rho(x,p,t) = \int \rho(x_0,p_0,0)\delta(x-x_t(x_0,p_0)) \delta(p-p_t(x_0,p_0)) \mathrm{d}x_0\mathrm{d}p_0
    \end{equation*}
    这里引入了$\delta$函数。$\delta$函数满足
    \begin{equation*}
        \delta(x-x_0) &= 0, \ \forall \ x \neq x_0\\
        \int_{-\infty}^{+\infty} \delta(x-x_0) \mathrm{d}x &= 1\\
        \int_{-\infty}^{+\infty} f(x)\delta(x-x_0) \mathrm{d}x &= f(x_0)
    \end{equation*}
    现在希望给$\delta$函数给一个形式,让它和上面满足的性质自洽:
    可以利用Fourier变换及其逆变换的定义
    \begin{equation*}
        \frac 1{\sqrt{2\pi}} \int_{-\infty}^{+\infty} f(x)\mathrm{e}^{\mathrm{i}kx}\mathrm{d}x &= F(k)\\
        \frac 1{\sqrt{2\pi}} \int_{-\infty}^{+\infty} F(k)\mathrm{e}^{-\mathrm{i}kx}\mathrm{d}k &= f(x)
    \end{equation*}
    于是有
    \begin{equation*}
        f(x_0) &= \frac 1{\sqrt{2\pi}} \int_{-\infty}^{+\infty} \frac 1{\sqrt{2\pi}} \int_{-\infty}^{+\infty} f(x)\mathrm{e}^{\mathrm{i}kx}\mathrm{d}x \mathrm{e}^{-\mathrm{i}kx_0}\mathrm{d}k\\
        &= \frac 1{2\pi} \iint f(x)\mathrm{e}^{\mathrm{i}k(x-x_0)}\mathrm{d}x\mathrm{d}k\\
        &= \frac 1{2\pi} \iint f(x)\mathrm{e}^{\mathrm{i}k(x-x_0)}\mathrm{d}k\mathrm{d}x
    \end{equation*}
    于是可以写出$\delta$函数为
    \begin{equation*}
        \delta(x-x_0) = \frac 1{2\pi} \int_{-\infty}^{+\infty} \mathrm{e}^{\mathrm{i}k(x-x_0)}\mathrm{d}k
    \end{equation*}

    某个物理量的期望定义为
    \begin{equation*}
        \langle B(t) \rangle = \int \rho(x,p,t) B(x,p) \mathrm{d}x\mathrm{d}p
    \end{equation*}
    回到用Morse势描述HCl的振动的问题,在这个问题下,初始时刻为Boltzmann分布时,
    \begin{equation*}
        \langle x \rangle &= 0\\
        \langle x^2 \rangle &= \frac 1{\beta m \omega^2}\\
        \Delta x &= \sqrt{\langle x^2 \rangle - \langle x \rangle ^2} = \frac 1{\sqrt{\beta m \omega^2}}
    \end{equation*}

    \section{Homework}
    \begin{asg}
        Boltzmann分布是否为稳态分布?
    \end{asg}
    \begin{asg}
        查阅\ce{H2}分子的红外光谱数据,构造\ce{H2}分子的Morse势表达式。
    \end{asg}
    \begin{asg}
        以Boltzmann分布为初始分布?,在Morse势,Euler图象下演化\ce{H2}的$t$时刻的分布。
    \end{asg}
    \begin{asg}
        以Boltzmann分布为初始分布?,在Morse势,Lagrange图象下演化\ce{H2}的$t$时刻的分布。
    \end{asg}

    \bibliographystyle{plain}
    \bibliography{ref_chp_3}