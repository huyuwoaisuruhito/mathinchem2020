\chapter{多自由度小振动}
    \section{将平衡位置附近的势能函数展开为二次型}
    现在研究复杂一些的\ce{H2O}分子的振动。它总共有3个原子,所以有9个运动自由度。质心平动3个自由度,
    刚性转动也有3个自由度,因此振动是3个自由度。
    \footnote{
    3个振动自由度分别为剪切振动、对称伸缩振动和不对称伸缩振动。
    水分子的O-H振动波数约为3700 cm$^{-1}$, 剪切振动波数约为1600 cm$^{-1}$, 
    伸缩振动1个周期应当约为20.8 fs, 剪切振动周期约为9 fs. 而1 a.u. = 0.024 fs.即可据此估计模拟过程中的时间步长。
    }
    对于水分子,定义每个原子的坐标为:
    \begin{equation}
        \begin{split}
        \bm{x} = 
        \begin{pmatrix}
            \bm{x}_\mathrm{O}\\
            \bm{x}_{\mathrm{H1}}\\
            \bm{x}_{\mathrm{H2}}
        \end{pmatrix}
    \end{split}
    \end{equation}
    其中$\bm{x}_\mr{O}$表示氧原子O在三维空间中的坐标,其他以此类推。
    给定原子核运动的势能$V(\bm{x})$, 定义质量矩阵:
    \begin{equation}
        \begin{split}
        \bm{M} = \mathrm{diag} \{m_1,...,m_9 \} = 
        \begin{pmatrix}
            m_1 & \cdots & 0\\
            \vdots & \ddots & \vdots\\
            0 & \cdots & m_9
        \end{pmatrix}
        \label{mass matrix}
    \end{split}
    \end{equation}
    其中,$m_1,m_2,m_3$等于氧原子的质量,$m_4,...,m_9$等于氢原子的质量。那么可以将
    系统的动量表示为:
    \begin{equation}
        \bm{p} = \bm{M}\cdot\bm{x}
    \end{equation}
    仿照一维系统的Hamilton量,可以写出这个多维系统的Hamilton量:
    \begin{equation}
        H(\bm{x}, \bm{p}) = \frac{1}{2}\bm{p}^{\mr{t}}\bm{M}^{-1}\bm{p} + V(\bm{x})
    \end{equation}
    理论上,只要给出势能函数的形式,就可以完全讨论系统的运动。但是对于系统在平衡位置
    \footnote{一般都是指稳定平衡位置,即当系统偏离平衡点的距离非常微小时,
    系统有回到平衡位置的运动趋势,数学上对应势能函数的\textbf{极小值点}
    }附近的运动,通常采用\textbf{小振动近似}来得到系统在平衡位置附近运动的解析表达式。
    假设平衡位置为$\bm{x_{\mr{eq}}}$,那么在平衡点邻域内的函数值可以按照Taylor展开写为:
    \begin{equation}
        \begin{split}
        V(\bm{x_{\mr{eq}}} + \bm{q}) &= V(\bm{x_{\mr{eq}}}) + \left.\pdv{V}{\bm{x}}\right|_{\bm{x} = \bm{x}_{\mr{eq}}}^{\mr{t}}\bm{q}
         + \frac{1}{2}\bm{q}^{\mr{t}}\left.\pdv{^2V}{\bm{x}^2}\right|_{\bm{x} = \bm{x_{\mr{eq}}}}\bm{q} + o(\left|\bm{q}\right|^2)\\
        \end{split}
    \end{equation}
    显然在平衡位置$\dps\left.\pdv{^2V}{\bm{x}^2}\right|_{\bm{x} = \bm{x}_{\mr{eq}}}$为0;同时常数项不会影响
    运动方程,可以不予考虑;如果在$\bm{q}$比较小时,忽略2阶以上的项\footnote{这就是小振动近似},那么就可以将势能函数
    重写为:
    \begin{equation}
        V(\bm{q}) = \frac{1}{2}\bm{q}^{\mr{t}}\left.\pdv{^2V}{\bm{x}^2}\right|_{\bm{x} = \bm{x}_{\mr{eq}}}\bm{q}
        \label{potential at equilibrium point}
    \end{equation}
    定义矩阵(通常而言是一个半正定
    \footnote{
        一般而言这个矩阵一定有0作为特征值。因为分子在某些
        自由度运动时(刚性转动、质心平动)分子的势能不变,这说明在势能极小值处
        沿着某些方向运动时势能函数恒定。可以说明代表分子整体平移的矢量是此矩阵特征值为0
        的特征向量,但是对于分子的转动,不一定会对应一个特征值为0的特征向量?
    }
    的实对称矩阵):
    \begin{equation}
        \bm{K} = \left.\pdv{^2V}{\bm{x}^2}\right|_{\bm{x} = \bm{x}_{\mr{eq}}}
        \label{potential matrix}
    \end{equation}
    那么系统在平衡位置的Hamilton量可以写为:
    \begin{equation}
        H(\bm{q}, \bm{p}) = \frac{1}{2}\bm{p}^{\mr{t}}\bm{M}^{-1}\bm{p} + \frac{1}{2}\bm{q}^{\mr{t}}\bm{K}\bm{q}
        \label{the Hamiltonian of samll vibration}
    \end{equation}
    其中$\bm{q}$为:
    \begin{equation}
        \bm{q} = \bm{x} - \bm{x_{\mr{eq}}}
    \end{equation}
    表示偏移平衡点的位移。
    \section{简正坐标}
    对于更一般的情况,质量矩阵$\bm{M}$不一定是对角的,但一定是实对称且正定的矩阵
    
    定义\textbf{Hessian}矩阵$\bm{\mathcal{H}}$为 
    \begin{equation*}
        \bm{\mathcal{H}}_{ij} = \frac 1{\sqrt{m_i}} \frac {\partial^2 V}{\partial x_i \partial x_j} \frac 1{\sqrt{m_j}}
    \end{equation*}
    它的单位为s$^{-2}$. 显然地,这是一个实对称矩阵,可以由正交矩阵作对角化:
    \begin{equation*}
        \bm{T}^\mathrm{T} \bm{\mathcal{H}T = \Omega}
    \end{equation*}
    其中$\bm{T}$为\textbf{正交矩阵},满足 
    \begin{equation*}
        \bm{T}^\mathrm{T}\bm{T} = \bm{TT}^\mathrm{T} = \bm{I}
    \end{equation*}
    并且
    \begin{equation*}
        \bm{\Omega} = \mathrm{diag} \{\omega_1^2, ..., \omega_9^2 \}
    \end{equation*}
    这样就得到了角频率,$\omega_j$对应的能量为$\hbar \omega_j$。总共得到了9个模式的频率,其中3个模式的频率对应平动,3个模式频率对应转动(平动转动的频率趋于0),3个模式频率对应振动。

    如果用矩阵形式来表示Hessian矩阵,应有
    \begin{equation*}
        \bm{\mathcal{H} = M}^{-\frac 12} \bm{V}^{(2)} \bm{M}^{-\frac 12}
    \end{equation*}
    上一节研究的Hessian矩阵对角化向量形式应为
    \begin{equation*}
        \bm{\mathcal{H}b}_j = \omega_j^2 \bm{b}_j
    \end{equation*}
    由此可知 
    \begin{equation*}
        \bm{H}
        \begin{pmatrix}
            \bm{b}_1 & \cdots & \bm{b}_N
        \end{pmatrix}
        = 
        \begin{pmatrix}
            \bm{b}_1 & \cdots & \bm{b}_N
        \end{pmatrix}
        \bm{\Omega}
    \end{equation*}
    所以
    \begin{equation*}
        \bm{T} = 
        \begin{pmatrix}
            \bm{b}_1 & \cdots & \bm{b}_N
        \end{pmatrix}
    \end{equation*}
    有
    \begin{equation*}
        \bm{HT = T\Omega}
    \end{equation*}
    要得到本征值,应当有
    \begin{equation*}
        \det (\bm{\mathcal{H}}- \omega^2 \bm{I}) = 0
    \end{equation*}
    即可得到$N$本征频率。
    \begin{asg}
        第3次作业第1题:证明Hessian矩阵的本征值都是实数。
    \end{asg}

    水分子在谐振子近似下的势能函数为
    \begin{align*}
        V(\bm{x}) &= V(\bm{x}_\mathrm{eq}) + \frac 12 (\bm{x-x}_\mathrm{eq})^\mathrm{T} \bm{V}^{(2)} (\bm{x-x}_\mathrm{eq})\\
        &= V(\bm{x}_\mathrm{eq}) + \frac 12 (\bm{x-x}_\mathrm{eq})^\mathrm{T} \bm{M}^{\frac 12}\bm{\mathcal{H}} \bm{M}^{\frac 12} (\bm{x-x}_\mathrm{eq})\\
        &= V(\bm{x}_\mathrm{eq}) + \frac 12 (\bm{x-x}_\mathrm{eq})^\mathrm{T} \bm{M}^{\frac 12} \bm{T\Omega T}^\mathrm{T} \bm{M}^{\frac 12} (\bm{x-x}_\mathrm{eq})
    \end{align*}
    定义\textbf{简正坐标}$\bm{Q}$为
    \begin{equation*}
        \bm{Q} = \bm{T}^\mathrm{T} \bm{M}^{\frac 12} (\bm{x-x}_\mathrm{eq})
    \end{equation*}
    于是势能面可以写为
    \begin{equation*}
        V(\bm{Q}) = V(\bm{0}) + \frac 12 \bm{Q}^\mathrm{T} \bm{\Omega Q} = V(\bm{0}) + \sum_{j=1}^N \frac 12 \omega_j^2 Q_j^2
    \end{equation*}

    \section{简正坐标和Cartesian坐标的关系}
    上一节讨论了势能在简正坐标下的形式。要想得到全能量,还需要给出动量在简正坐标下的形式。根据
    \begin{equation*}
        \bm{p} = \bm{M\dot{x}} = \bm{M}^{\frac 12} \bm{T\dot{Q}}
    \end{equation*}
    这是在Cartesian坐标系下的动量。定义简正坐标下的动量为
    \begin{equation*}
        \bm{P = \dot{Q}} = \bm{T}^\mathrm{T} \bm{M}^{-\frac 12} \bm{p}
    \end{equation*}
    那么,可以得到动能的表达式为
    \begin{align*}
        E_\mathrm{k} = \frac 12 \bm{p}^{\mathrm{T}} \bm{M}^{-1} \bm{p} &= \frac 12 \bm{\dot{x}}^\mathrm{T}\bm{M\dot{x}}
        = \frac 12 \bm{\dot{Q}}^\mathrm{T} \bm{Q}
        = \frac 12 \bm{P}^{\mathrm{T}} \bm{P}
    \end{align*}
    总结简正坐标和Cartesian坐标的变换:
    \begin{align*}
        \bm{Q} &= \bm{T}^\mathrm{T} \bm{M}^{\frac 12} (\bm{x-x}_\mathrm{eq})\\
        \bm{P} &= \bm{T}^\mathrm{T} \bm{M}^{-\frac 12} \bm{p}\\
        \bm{x} &= \bm{x}_\mathrm{eq} + \bm{M}^{-\frac 12}\bm{TQ}\\
        \bm{p} &= \bm{M}^{\frac 12}\bm{TP}
    \end{align*}

    有了简正坐标下的动量就可以得到简正坐标下的Hamilton函数:
    \begin{equation*}
        H = \frac 12 \bm{P}^\mathrm{T}\bm{P} + \frac 12 \bm{Q}^\mathrm{T} \bm{\Omega Q}
    \end{equation*}
    很容易可以验证,正则方程在简正坐标下依旧成立:
    \begin{align*}
        \bm{\dot{Q}} &= \frac {\partial H}{\partial \bm{P}} = \bm{P}\\
        \bm{\dot{P}} &= -\frac {\partial H}{\partial \bm{Q}} = -\bm{\Omega Q}
    \end{align*}
    要求出每个元素的值也十分容易:
    \begin{align*}
        \dot{Q}_j &= P_j\\
        \dot{P}_j &= - \omega_j^2 Q_j
    \end{align*}
    但如果在Cartesian坐标下用正则方程,得到每个元素的值结果为
    \begin{align*}
        \dot{x}_i &= \frac {p_j}{m_i}\\
        \dot{p}_i &= \sum_j \frac {\partial^2 V}{\partial x_i \partial x_j} (x_j - x_\mathrm{eq}^{(j)})
    \end{align*}
    显然要比在简正坐标下的形式要复杂很多。这体现了简正坐标的优势。

    我们可以根据初始条件$(\bm{x}_0,\bm{p}_0)$,得到$(\bm{x},\bm{p}$每个分量的解析表达式。将它变换为简正坐标,得到
    \begin{align*}
        Q_j &= \sum_i T_{ij} m_i^{\frac 12} (x_0^{(i)} - x_\mathrm{eq}^{(i)})\\
        P_j &= \sum_i T_{ij} m_i^{-\frac 12} (p_0^{(i)})
    \end{align*}
    由正则方程的形式可以给出
    \begin{align*}
        Q_j(t) &= Q_0^{(j)}\cos{\omega_j t} + \frac {P_0^{(j)}}{\omega} \sin{\omega_j t}\\
        P_j(t) &= P_0^{(j)}\cos{\omega_j t} - \omega_j Q_0^{(j)} \sin{\omega_j t}
    \end{align*}
    再变换回Cartesian坐标,得到
    \begin{align*}
        x_i(t) - x_\mathrm{eq}^{(i)} &= \sum_j m_j^{-\frac 12} T_{ij} Q_j(t)\\
        p_i(t) &= \sum_j m_j^{\frac 12} T_{ij}P_j(t)
    \end{align*}

    如果水分子服从Boltzmann分布,即
    \begin{align*}
        \rho \propto \mathrm{e}^{-\beta (\frac 12 \bm{P}^\mathrm{T}\bm{P} + \frac 12 \bm{Q}^\mathrm{T} \bm{\Omega Q})} = \mathrm{e}^{- \frac {\beta}2 \sum_j (P_j^2 + \omega_j^2 Q_j^2)} = \prod_j \mathrm{e}^{-\frac {\beta}2 (P_j^2 + \omega_j^2 Q_j^2)}
    \end{align*}
    要想将Cartesian坐标下的积分变换成简正坐标下的积分,需要计算Jacobi行列式:
    \begin{align*}
        \bigg|\frac {\partial (\bm{Q,P})}{\partial (\bm{x,p})}\bigg| &= \det
        \begin{pmatrix}
            \bm{T}^\mathrm{T}\bm{M}^{\frac 12} & 0\\
            0 & \bm{T}^\mathrm{T}\bm{M}^{-\frac 12}
        \end{pmatrix}
        = 1
    \end{align*}
    这里用到了正交矩阵的行列式为1(这也是显然成立的)。因此可以积分得到配分函数,从而得到简正坐标下的某个分量概率密度为
    \begin{equation*}
        \mathcal{P}_j = \frac {2\pi}{\beta\omega_j} \mathrm{e}^{-\frac {\beta}2 (P_j^2 + \omega_j^2 Q_j^2)}
    \end{equation*}
    总概率密度为
    \begin{equation*}
        \mathcal{P} = \prod_j \mathcal{P}_j
    \end{equation*}
    由此可以得到Cartesian坐标下概率密度。

    \section{Homework}
    \begin{asg}
        水分子的简谐振动分析
    \end{asg}


    \bibliographystyle{plain}
    \bibliography{ref_chp_4}
