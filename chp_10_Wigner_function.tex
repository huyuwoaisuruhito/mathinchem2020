\chapter{Winger 函数}
    \section{Winger 函数的定义与性质}

        根据经典情况统计物理的结果
        \begin{equation*}
            \langle A \rangle = \int \rho(x,p) A(x,p) \mathrm{d}x\mathrm{d}p
        \end{equation*}
        其中 
        \begin{align*}
            \rho &= \frac 1Z \mathrm{e}^{-\beta H(x,p)}\\
            Z &= \int \frac 1{2\pi \hbar} \mathrm{e}^{-\beta H(x,p)} \mathrm{d}x \mathrm{d}p
        \end{align*}
        现在要问,量子力学有没有同样的形式?

        定义算符的Winger密度分布函数
        \begin{align*}
            A_\mathrm{w}(x,p) &= \int \langle x - \frac {\Delta}2 | \hat{A} | x + \frac {\Delta}2 \rangle \mathrm{e}^{\frac {\mathrm{i} p \Delta}{\hbar}} \mathrm{d} \Delta \\
            B_\mathrm{w}(x,p) &= \int \langle x - \frac {\Delta}2 | \hat{B} | x + \frac {\Delta}2 \rangle \mathrm{e}^{\frac {\mathrm{i} p \Delta}{\hbar}} \mathrm{d} \Delta 
        \end{align*}
        可以验证,如果$\hbar \to 0$, 应有$A_\mathrm{w} \to A_\mathrm{cl}$, $B_\mathrm{w} \to B_\mathrm{cl}$。且可以验证
        \begin{align*}
            \mathrm{Tr} (\hat{A}\hat{B}) &= \sum_{ij} \langle \phi_{i} | \hat{A} | \phi_{j} \rangle \langle \phi_{j} | \hat{B} | \phi_{i} \rangle \\
            &= \frac 1{2\pi\hbar} \int A_\mathrm{w}(x,p) B_\mathrm{w}(x,p) \mathrm{d}x \mathrm{d}p
        \end{align*}

        定义密度算符
        \begin{equation*}
            \hat{\rho} = |\psi \rangle \langle \psi |
        \end{equation*}
        可以在位置空间和动量空间给出概率密度分布。并可以验证
        \begin{align*}
            \int \rho_\mathrm{w}(x,p) \mathrm{d}x &= \langle p|\hat{\rho}|p \rangle\\
            \int \rho_\mathrm{w}(x,p) \mathrm{d}p &= \langle x|\hat{\rho}|x \rangle
        \end{align*}
        物理量期望值可以写成
        \begin{align*}
            \langle \psi | \hat{B} | \psi \rangle &=  \langle \psi |\sum_n | \phi_n \rangle \langle \phi_n | \hat{B} | \psi \rangle\\
            &= \sum_n \langle \phi_n | \hat{B} | \psi \rangle \langle \psi  | \phi_n \rangle\\
            &= \mathrm{Tr} (\hat{B}|\psi \rangle \langle \psi |)
        \end{align*}
        由前文的讨论可以发现
        \begin{equation*}
            \langle \psi|\hat{B}|\psi \rangle = \frac 1{2\pi \hbar} \int \rho_\mathrm{w}(x,p) B_\mathrm{w}(x,p) \mathrm{d}x\mathrm{d}p
        \end{equation*}
        \bibliographystyle{plain}
    \bibliography{ref_chp_10}